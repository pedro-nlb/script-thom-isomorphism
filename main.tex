\documentclass[12pt,a4paper]{amsart}

\usepackage[T1]{fontenc}
\usepackage[utf8]{inputenc}
\usepackage[british]{babel}
\usepackage{mathtools}
\usepackage{amsthm}
\usepackage{amssymb}
\usepackage{mathrsfs}
\usepackage{enumitem}
\usepackage{tikz-cd}
\usetikzlibrary{decorations.markings}
\usepackage{float}
\usepackage{hyperref}
\urlstyle{same}
\usepackage[noabbrev]{cleveref}

\theoremstyle{plain}
\newtheorem{thm}{Theorem}
\newtheorem*{thm*}{Theorem}
\newtheorem{lm}[thm]{Lemma}
\newtheorem{prop}[thm]{Proposition}
\newtheorem{cor}[thm]{Corollary}
\theoremstyle{definition}
\newtheorem{defn}[thm]{Definition}
\newtheorem{exmp}[thm]{Example}
\newtheorem{xca}[thm]{Exercise}
\theoremstyle{remark}
\newtheorem{rem}[thm]{Remark}
\Crefname{thm}{Theorem}{Theorems}
\Crefname{lm}{Lemma}{Lemmas}
\Crefname{prop}{Proposition}{Propositions}
\Crefname{cor}{Corollary}{Corollaries}
\Crefname{defn}{Definition}{Definitions}
\Crefname{exmp}{Example}{Examples}
\Crefname{xca}{Exercise}{Exercises}
\Crefname{rem}{Remark}{Remarks}

\title[The Thom isomorphism]{The Thom isomorphism}
\author[Pedro N\'{u}\~{n}ez]{Pedro N\'{u}\~{n}ez}
\address{Pedro N\'{u}\~{n}ez \newline
\indent Albert-Ludwigs-Universit\"{a}t Freiburg, Mathematisches Institut \newline
\indent Ernst-Zermelo-Straße 1, 79104 Freiburg im Breisgau (Germany)}
\email{\normalfont\href{mailto:pedro.nunez@math.uni-freiburg.de}{pedro.nunez@math.uni-freiburg.de}}
\renewcommand*{\urladdrname}{\itshape Homepage}
\urladdr{\normalfont\href{https://home.mathematik.uni-freiburg.de/nunez/}{https://home.mathematik.uni-freiburg.de/nunez}}
\thanks{The author gratefully acknowledges support by the DFG-Graduiertenkolleg GK1821 ``Cohomological Methods in Geometry'' at the University of Freiburg.}
\date{\today}

\setcounter{tocdepth}{1}
\sloppy
\makeatletter
\hypersetup{
  pdfauthor={\authors},
  pdftitle={\@title},
  colorlinks,
  linkcolor=[rgb]{0.2,0.2,0.6},
  citecolor=[rgb]{0.2,0.6,0.2},
  urlcolor=[rgb]{0.6,0.2,0.2}}
\makeatother
\makeatletter
\tikzcdset{
  open/.code={\tikzcdset{hook, circled};},
  closed/.code={\tikzcdset{hook, slashed};},
  circled/.code={\tikzcdset{markwith={\draw (0,0) circle (.375ex);}};},
  slashed/.code={\tikzcdset{markwith={\draw[-] (-.4ex,-.4ex) -- (.4ex,.4ex);}};},
  markwith/.code={
    \pgfutil@ifundefined{tikz@library@decorations.markings@loaded}%
    {\pgfutil@packageerror{tikz-cd}{You need to say %
    \string\usetikzlibrary{decorations.markings} to use arrow with markings}{}}{}%
    \pgfkeysalso{/tikz/postaction={/tikz/decorate,
    /tikz/decoration={
      markings,
      mark = at position 0.5 with
      {#1}}}}},
}
\makeatother

\begin{document}

\maketitle

\begin{abstract}
  Script for a talk of the Wednesday Seminar of the GK1821 at Freiburg during the Summer Semester 2021.
  The main reference is \cite[\S 2]{ati67}.
\end{abstract}

\tableofcontents

\begin{center}
  \textcolor{gray}{---parts in gray will be omitted during the talk---}
\end{center}

\section{Setting and conventions}

\begin{itemize}
  \item We work with complex vector spaces and complex vector bundles only \cite[p.~1]{ati67}.
  \item We use the usual word \textit{rank} instead of \textit{dimension}, which is the one used in \cite[p.~3]{ati67}.
  \item All base spaces are implicitly assumed to be compact and Hausdorff, although reminders will appear now and then.
    The usual notation for a base space will be $X$.
  \item We use $\operatorname{Vect}(X)$ to denote the set of isomorphism classes of vector bundles $X$, and $\operatorname{Vect}_{n}(X)$ to denote the subset of $\operatorname{Vect}(X)$ given by bundles of rank $n$ \cite[p.~17]{ati67}.
    Note that $(\operatorname{Vect}(X), \oplus)$ is a commutative monoid \cite[p.~17]{ati67}, and $(\operatorname{Vect}(X), \oplus, \otimes)$ is a semiring.
  \item Given a commutative monoid $A$, we denote by $K(A)$ its Grothendieck group \cite[p.~42]{ati67}.
    If $A$ is also a semiring, we regard $K(A)$ as a ring with the induced ring structure \cite[p.~43]{ati67}.
  \item We denote $K(X) := K(\operatorname{Vect}(X))$, which is then a commutative ring with one \cite[p.~43]{ati67}.
    We think of elements of $K(X)$ as formal differences $[E] - [F]$ of vector bundles $E$ and $F$ on $X$ \cite[p.~44]{ati67}.
  \item We write $\underline{n}$ for the trivial bundle of rank $n$.
  \item We denote by $\mathcal{C}$ the category of compact topological spaces, by $\mathcal{C}^{+}$ the category of compact spaces with distinguished basepoints and by $\mathcal{C}^{2}$ the category of compact pairs.
\end{itemize}

\section{Recollection from previous talks}

We have a functor $\mathcal{C}^{2} \to \mathcal{C}^{+}$ that sends a pair $(X,Y)$ to $X/Y$, with basepoint $Y/Y$.
If $Y = \varnothing$, then we interpret the resulting object as $X$ with a disjoint basepoint.
We also have a functor $\mathcal{C} \to \mathcal{C}^{2}$ sending $X \mapsto (X, \varnothing)$.
Hence, the composition of the two functors gives $X \mapsto X^{+}$, where $X^{+}$ is the disjoint union of $X$ with a basepoint.

For $X$ in $\mathcal{C}^{+}$ we define $\tilde{K}(X)$ to be the kernel of the map $i^{*} \colon K(X) \to K(x_{0})$, where $i \colon x_{0} \to X$ is the inclusion of the basepoint.
If $c \colon X \to x_{0}$ is the collapsing map, then $c^{*}$ induces a splitting
\[ K(X) \cong \tilde{K}(X) \oplus K(x_{0}). \]
Indeed, we need...

\appendix

\section{More on conventions and preliminaries}

\subsection{Construction of $K(X)$}

The Grothendieck group is defined via universal property, but let us agree on a specific construction in order to have a precise description of the elements in the ring $K(X)$.
We follow both \cite{ati67} and \cite{hat03} and consider the construction $1$ described by Jin in the first talk of the seminar, which is the second construction discussed by Atiyah in \cite[p.~42]{ati67}.

\begin{lm}
  Let $M$ be a commutative monoid.
  Then Jin's construction $1$ agrees with Atiyah's second construction of $K(M)$.
  \begin{proof}
    In both cases $K(M)$ is the quotient of $M \times M$ by an equivalence relation, so it suffices to show that the equivalence relations agree.
    In Atiyah's construction we have
    \[ (x,y) \sim_{A} (x', y') : \Leftrightarrow \exists z , z' \in M,, (x + z, y + z) = (x' + z', y' + z'). \]
    In Jin's construction we have
    \[ (x,y) \sim_{J} (x', y') : \Leftrightarrow \exists z \in M,, x + y' + z = x' + y + z. \]
    If $(x, y) \sim_{A} (x', y')$, then we have $x + z = x' + z'$ and $y + z = y' + z'$ for some $z, z' \in M$. 
    Associativity and commutativity of $M$ imply that
    \[ x + y' + z + z' = x' + y' + z + z' = x' + y + z + z', \]
    hence $(x, y) \sim_{J} (x', y')$.
    Conversely, if $(x, y) \sim_{J} (x', y')$, then we have $x + y' + z = x' + y + z$ for some $z \in M$.
    In particular we have
    \begin{align*}
      (x + (x + y' + z), y + (x + y' + z)) & = (x + (x' + y + z), y' + (x + y + z)) \\
      & = (x' + (x + y + z), y' + (x + y + z)),
    \end{align*}
    so $(x, y) \sim_{A} (x', y')$ as well.
  \end{proof}
\end{lm}

Given (an isomorphism class of) a vector bundle $E \in \operatorname{Vect}(X)$, we denote by $[E]$ its image in $K(X)$, that is, $[E] = [(E, 0)]$.
Since $-[E] = [(0,E)]$, we can write every element $[(E,F)] \in K(X)$ as $[E] - [F]$.
We can find some vector bundle $G$ such that $F \oplus G$ is trivial \cite[Corollary 1.4.14]{ati67}.
With the notation introduced earlier we can write $[F \oplus G] = [\underline{n}]$ for some $n \in \mathbb{N}$.
Then we would have
\[ [E] - [F] = [E] + [G] - ([F] + [G]) = [E \oplus G] - [\underline{n}], \]
showing that every element of $K(X)$ can be written as $[H] - [\underline{n}]$ for some vector bundle $H$ on $X$ and some natural number $n \in \mathbb{N}$ \cite[p.~44]{ati67}.

Suppose now that $E$ and $F$ are such that $[E] = [F]$, that is, $[(E,0)] = [(F,0)]$.
By definition of the equivalence relation that we are using, there exists some vector bundle $G$ such that $E \oplus G \cong F \oplus G$.
Applying \cite[Corollary 1.4.14]{ati67} again we deduce that $E \oplus \underline{n} \cong F \oplus \underline{n}$ for some $n \in \mathbb{N}$.
In this case we say that $E$ and $F$ are \textit{stably equivalent}.
This brings us to Hatcher's description of $K(X)$ \cite[p.~39]{hat03}, namely, as formal differences $E - E'$ in which we identify $E_{1} - E_{1}'$ with $E_{2} - E_{2}'$ if and only if $E_{1} \oplus E_{2}'$ and $E_{2} \oplus E_{1}'$ are stably equivalent, that is, if and only if $[E_{1} \oplus E_{2}'] = [E_{2} \oplus E_{1}']$.
Since $[E_{1} \oplus E_{2}'] = [E_{1}] + [E_{2}']$ and $[E_{2} \oplus E_{1}'] = [E_{2}] + [E_{1}']$, we do have $E_{1} - E_{1}' = E_{2} - E_{2}'$ in Hatcher's sense if and only if $[(E_{1},E_{1}')] = [(E_{2},E_{2}')]$ in Atiyah's sense.
We will try to follow Atiyah's notation most of the time.

\bibliographystyle{alpha}
\bibliography{main}
\vfill

\end{document}
