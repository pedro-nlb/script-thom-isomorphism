\documentclass[12pt,a4paper]{amsart}

\usepackage[T1]{fontenc}
\usepackage[utf8]{inputenc}
\usepackage[british]{babel}
\usepackage{mathtools}
\usepackage{amsthm}
\usepackage{amssymb}
\usepackage{mathrsfs}
\usepackage{enumitem}
\usepackage{tikz-cd}
\usetikzlibrary{decorations.markings}
\usepackage{float}
\usepackage{hyperref}
\urlstyle{same}
\usepackage[noabbrev]{cleveref}

\theoremstyle{plain}
\newtheorem{thm}{Theorem}
\newtheorem*{thm*}{Theorem}
\newtheorem{lm}[thm]{Lemma}
\newtheorem{prop}[thm]{Proposition}
\newtheorem{cor}[thm]{Corollary}
\theoremstyle{definition}
\newtheorem{defn}[thm]{Definition}
\newtheorem{exmp}[thm]{Example}
\newtheorem{xca}[thm]{Exercise}
\theoremstyle{remark}
\newtheorem{rem}[thm]{Remark}
\Crefname{thm}{Theorem}{Theorems}
\Crefname{lm}{Lemma}{Lemmas}
\Crefname{prop}{Proposition}{Propositions}
\Crefname{cor}{Corollary}{Corollaries}
\Crefname{defn}{Definition}{Definitions}
\Crefname{exmp}{Example}{Examples}
\Crefname{xca}{Exercise}{Exercises}
\Crefname{rem}{Remark}{Remarks}

\title[The Thom isomorphism]{The Thom isomorphism}
\author[Pedro N\'{u}\~{n}ez]{Pedro N\'{u}\~{n}ez}
\address{Pedro N\'{u}\~{n}ez \newline
\indent Albert-Ludwigs-Universit\"{a}t Freiburg, Mathematisches Institut \newline
\indent Ernst-Zermelo-Straße 1, 79104 Freiburg im Breisgau (Germany)}
\email{\normalfont\href{mailto:pedro.nunez@math.uni-freiburg.de}{pedro.nunez@math.uni-freiburg.de}}
\renewcommand*{\urladdrname}{\itshape Homepage}
\urladdr{\normalfont\href{https://home.mathematik.uni-freiburg.de/nunez/}{https://home.mathematik.uni-freiburg.de/nunez}}
\thanks{The author gratefully acknowledges support by the DFG-Graduiertenkolleg GK1821 ``Cohomological Methods in Geometry'' at the University of Freiburg.}
\date{\today}

\setcounter{tocdepth}{1}
\sloppy
\makeatletter
\hypersetup{
  pdfauthor={\authors},
  pdftitle={\@title},
  colorlinks,
  linkcolor=[rgb]{0.2,0.2,0.6},
  citecolor=[rgb]{0.2,0.6,0.2},
  urlcolor=[rgb]{0.6,0.2,0.2}}
\makeatother
\makeatletter
\tikzcdset{
  open/.code={\tikzcdset{hook, circled};},
  closed/.code={\tikzcdset{hook, slashed};},
  circled/.code={\tikzcdset{markwith={\draw (0,0) circle (.375ex);}};},
  slashed/.code={\tikzcdset{markwith={\draw[-] (-.4ex,-.4ex) -- (.4ex,.4ex);}};},
  markwith/.code={
    \pgfutil@ifundefined{tikz@library@decorations.markings@loaded}%
    {\pgfutil@packageerror{tikz-cd}{You need to say %
    \string\usetikzlibrary{decorations.markings} to use arrow with markings}{}}{}%
    \pgfkeysalso{/tikz/postaction={/tikz/decorate,
    /tikz/decoration={
      markings,
      mark = at position 0.5 with
      {#1}}}}},
}
\makeatother

\begin{document}

\maketitle

\begin{abstract}
  Script for a talk of the Wednesday Seminar of the GK1821 at Freiburg during the Summer Semester 2021.
  The main reference is \cite[\S 2]{ati67}.
\end{abstract}

\tableofcontents

\begin{center}
  \textcolor{gray}{---parts in gray will be omitted during the talk---}
\end{center}

\section{Setting and conventions}

\begin{itemize}
  \item We work with complex vector spaces and complex vector bundles only.
  \item We use the usual word \textit{rank} instead of \textit{dimension}, which is the one used in \cite{ati67}.
  \item We use $\operatorname{Vect}(X)$ to denote the set of isomorphism classes of vector bundles on $X$, and $\operatorname{Vect}_{n}(X)$ to denote the subset of $\operatorname{Vect}(X)$ given by bundles of dimension $n$.
  \item We denote by $\mathcal{C}$ the category of compact spaces, by $\mathcal{C}^{+}$ the category of compact spaces with distinguished basepoints and by $\mathcal{C}^{2}$ the category of compact pairs.
\end{itemize}

\section{Recollection from previous talks}

\subsection{Definition of $K(X)$}

If $X$ is any space, the set $\operatorname{Vect}(X)$ has the structure of an abelian semigroup, where the additive structure is defined by direct sum.

We have a functor $\mathcal{C}^{2} \to \mathcal{C}^{+}$ that sends a pair $(X,Y)$ to $X/Y$, with basepoint $Y/Y$.
If $Y = \varnothing$, then we interpret the resulting object as $X$ with a disjoint basepoint.
We also have a functor $\mathcal{C} \to \mathcal{C}^{2}$ sending $X \mapsto (X, \varnothing)$.
Hence, the composition of the two functors gives $X \mapsto X^{+}$, where $X^{+}$ is the disjoint union of $X$ with a basepoint.

For $X$ in $\mathcal{C}^{+}$ we define $\tilde{K}(X)$ to be the kernel of the map $i^{*} \colon K(X) \to K(x_{0})$, where $i \colon x_{0} \to X$ is the inclusion of the basepoint.
If $c \colon X \to x_{0}$ is the collapsing map, then $c^{*}$ induces a splitting
\[ K(X) \cong \tilde{K}(X) \oplus K(x_{0}). \]
Indeed, we need...

\bibliographystyle{alpha}
\bibliography{main}
\vfill

\end{document}
