\documentclass[12pt,a4paper]{amsart}

\usepackage[T1]{fontenc}
\usepackage[utf8]{inputenc}
\usepackage[british]{babel}
\usepackage{mathtools}
\usepackage{amsthm}
\usepackage{amssymb}
\usepackage{mathrsfs}
\usepackage{enumitem}
\usepackage{tikz-cd}
\usetikzlibrary{decorations.markings}
\usepackage{float}
\usepackage{hyperref}
\urlstyle{same}
\usepackage[noabbrev]{cleveref}

\theoremstyle{plain}
\newtheorem{thm}{Theorem}
\newtheorem*{thm*}{Theorem}
\newtheorem{lm}[thm]{Lemma}
\newtheorem{prop}[thm]{Proposition}
\newtheorem{cor}[thm]{Corollary}
\theoremstyle{definition}
\newtheorem{defn}[thm]{Definition}
\newtheorem{exmp}[thm]{Example}
\newtheorem{xca}[thm]{Exercise}
\theoremstyle{remark}
\newtheorem{rem}[thm]{Remark}
\Crefname{thm}{Theorem}{Theorems}
\Crefname{lm}{Lemma}{Lemmas}
\Crefname{prop}{Proposition}{Propositions}
\Crefname{cor}{Corollary}{Corollaries}
\Crefname{defn}{Definition}{Definitions}
\Crefname{exmp}{Example}{Examples}
\Crefname{xca}{Exercise}{Exercises}
\Crefname{rem}{Remark}{Remarks}

\title[The Thom isomorphism]{The Thom isomorphism}
\author[Pedro N\'{u}\~{n}ez]{Pedro N\'{u}\~{n}ez}
\address{Pedro N\'{u}\~{n}ez \newline
\indent Albert-Ludwigs-Universit\"{a}t Freiburg, Mathematisches Institut \newline
\indent Ernst-Zermelo-Straße 1, 79104 Freiburg im Breisgau (Germany)}
\email{\normalfont\href{mailto:pedro.nunez@math.uni-freiburg.de}{pedro.nunez@math.uni-freiburg.de}}
\renewcommand*{\urladdrname}{\itshape Homepage}
\urladdr{\normalfont\href{https://home.mathematik.uni-freiburg.de/nunez/}{https://home.mathematik.uni-freiburg.de/nunez}}
\thanks{The author gratefully acknowledges support by the DFG-Graduiertenkolleg GK1821 ``Cohomological Methods in Geometry'' at the University of Freiburg.}
\date{30th June 2021}

\setcounter{tocdepth}{1}
\sloppy
\makeatletter
\hypersetup{
  pdfauthor={\authors},
  pdftitle={\@title},
  colorlinks,
  linkcolor=[rgb]{0.2,0.2,0.6},
  citecolor=[rgb]{0.2,0.6,0.2},
  urlcolor=[rgb]{0.6,0.2,0.2}}
\makeatother

\begin{document}

\maketitle

\begin{abstract}
  Script for a talk of the Wednesday Seminar of the GK1821 at Freiburg during the Summer Semester 2021.
  The main reference is \cite[\S 2]{ati67}.
\end{abstract}

\tableofcontents

\section{Recollections from previous talks}

Let us start by recalling the basic definitions and the key results that we have seen in previous talks.
We are only going to give a sketchy overview here; we refer to \cite{ati67} for the missing details and for most of the relevant conventions and notation.

\subsection{$K$-rings}

Let $X$ be a compact Hausdorff space.
We have defined its \textit{$K$-group} $K(X)$ as the Grothendieck group completion of the commutative monoid of isomorphism classes of complex vector bundles on $X$ with direct sum as addition.
There are various ways to construct $K(X)$; let us fix one explicit construction for concreteness.
Elements in $K(X)$ are equivalence classes of pairs $(E,F)$ with $E$ and $F$ (isomorphism classes of) vector bundles on $X$.
Two such pairs $(E,F)$ and $(E',F')$ define the same equivalence class if and only if there exists vector bundles $G$ and $G'$ such that $(E \oplus G, F \oplus G) = (E' \oplus G', F' \oplus G')$.
We denote by $[E]$ the equivalence class $[(E,X \times \{ 0\})]$, so that $[(E,F)] = [E] - [F]$.
The tensor product of vector bundles induces a product in $K(X)$ which turns it into a commutative ring with $1 = [(X \times \mathbb{C}, X \times \{ 0 \})]$.
The pullback of vector bundles along continuous functions induces morphisms of unital rings, so that $K(-)$ is a contravariant functor from the category of compact Hausdorff spaces to the category of commutative unital rings.

\begin{exmp}
  Let $X = \{ * \}$ be a point.
  Then $K(\{ * \}) \cong \mathbb{Z}$ as rings, because the semiring of isomorphism classes of vector bundles on $\{ * \}$ is isomorphic to the natural numbers, each vector bundle corresponding to its rank.
\end{exmp}

\subsection{Reduced $K$-groups}

Let now $X$ be a compact Hausdorff space with a basepoint $x_{0} \in X$.
Then we have defined its \textit{reduced $K$-group} $\tilde{K}(X)$ as the kernel of the ring morphism $i^{*} \colon K(X) \to K(\{ x_{0} \})$.
Thus $[(E,F)]\in \tilde{K}(X)$ if and only if $\dim(E_{x_{0}}) = \dim(F_{x_{0}})$, because the difference $\dim(E_{x_{0}}) - \dim(F_{x_{0}})$ only depends on the equivalence class $[(E,F)]$.
Since $(f^{*}E)_{y} = E_{f(y)}$, the pullback of vector bundles along continuous functions induces morphisms between the reduced $K$-groups, so that $\tilde{K}(-)$ is a contravariant functor from the category of comapct Hasudorff spaces to the category of abelian groups\footnote{In fact, $\tilde{K}(-)$ takes values in non-unital commutative rings, because each $\tilde{K}(X)$ is an ideal of the ring $K(X)$.
But its ring structure is often not that interesting.
For example, every element in $\tilde{K}(X)$ is nilpotent if $X$ is connected \cite[II.5.9]{kar78}.
If moreover $X$ is a union of two compact contractible subspaces containing the basepoint, then the product in $\tilde{K}(X)$ is trivial \cite[Example 2.13]{hat03}.}.

\begin{lm}
  Let $X$ be a compact Hausdorff space; let $i \colon \{ x_{0} \} \to X$ be the inclusion of a basepoint and let $c \colon X \to \{ x_{0} \}$ be the morphism contracting $X$ to the basepoint.
  Then $c^{*}$ induces a natural splitting of the short exact sequence
  \[ 0 \to \tilde{K}(X) \to K(X) \to K(\{ x_{0}\}) \to 0. \]
  In particular, we have a group\footnote{It cannot be an isomorphism of rings in general becuase $\tilde{K}(X)$ does not have a unit in general.} isomorphism
  \begin{align*}
    K(X) & \cong \tilde{K}(X) \oplus K(\{ x_{0} \}) \\
    \xi & \mapsto (\xi - c^{*}i^{*}\xi, i^{*}\xi)
  \end{align*}
  \begin{proof}
    See \cite[p.~66]{ati67}.
  \end{proof}
\end{lm}

\subsection{Relative $K$-groups}

Let $X$ be a compact Hausdorff space and $A \subseteq X$ a closed subspace.
Then we define the \textit{relative $K$-group} $K(X,A)$ to be $\tilde{K}(X/A)$, where we think of $X/A$ as $X \sqcup_{A} \{ * \}$, so that $X/\varnothing = X^{+}$ is the result of adding a disjoint basepoint to $X$.
Again, the resulting functor $K(-,-)$ takes values in the category of non-unital commutative rings, but we often just think of it as taking values in the category of abelian groups.

\subsection{Relation between the different $K$-groups}

They are related as follows:

\begin{lm}
  Let $X$ be a compact Hausdorff space.
  Then there are canonical ring isomorphisms
  \[ K(X) \cong \tilde{K}(X^{+}) = K(X,\varnothing). \]
  \begin{proof}
    The equality on the right hand side follows from the equality at the level of spaces, which was already mentioned above.
    The ring isomorphism on the left hand side is given by
    \[ [(E,F)] \mapsto [(E \sqcup (\{ * \} \times \{ 0 \}), F \sqcup (\{ * \} \times \{ 0 \}))]. \]
    Direct computation shows that this is indeed a ring morphism.
  \end{proof}
\end{lm}

\subsection{Negative $K$-groups}

Let $X$ be a compact Hausdorff space, let $n \in \mathbb{N}$ a natural number and let $x_{0} \in X$ be a basepoint.
Then we define $\tilde{K}^{-n}(X) := \tilde{K}(S^{n}X)$.
Similarly, for a closed subspace $A \subseteq X$ we define $K^{-n}(X,A) := \tilde{K}^{-n}(X/A)$.
And finally, in order to keep the same relations between the $K$-functors as in the case of $n = 0$, we just define $K^{-n}(X) := K^{-n}(X,\varnothing)$.
Once again, we think of the resulting functors as only taking values in abelian groups.
In this case we have a more relevant reason to do so.
Namely, that we are later going to define a product on the direct sum $\oplus_{n \in \mathbb{N}} K^{-n}(X)$ turning it into a graded ring and extending the previous ring structure on $K^{0}(X)$.

\subsection{The long exact sequence of a pair}

Let $X$ be a compact Hausdorff space and let $A \subseteq X$ be a closed subspace.
We have seen in Vera's talk \cite[Proposition 2.4.4]{ati67} that there is an exact sequence

\[ \cdots \to K^{-1}(A) \to K^{0}(X,A) \to K^{0}(X) \to K^{0}(A). \]

As an immediate consequence \cite[Corollary 2.4.7]{ati67}, if $i \colon A \hookrightarrow X$ is a retract of $X$, i.e.~if there exists $r \colon X \to A$ such that $r \circ i = \operatorname{id}_{A}$, then the exact sequence splits and we obtain isomorphisms $K^{-n}(X) \cong K^{-n}(X,A) \oplus K^{-n}(A)$ for all $n \in \mathbb{N}$.
If $Y$ is another compact Hausdorff space and we choose basepoints both on $X$ and on $Y$, we can apply this as in \cite[Corollary 2.4.8]{ati67} to obtain isomorphisms

\[ \tilde{K}^{-n}(X \times Y) \cong \tilde{K}^{-n}(X \wedge Y) \oplus \tilde{K}^{-n}(X) \oplus \tilde{K}^{-n}(Y) \]

for all $n \in \mathbb{N}$.

\subsection{External products}

For compact Hausdorff spaces $X$ and $Y$ and vector bundles $E$ and $F$ on $X$ and $Y$ respectively, we denote
\[ E \boxtimes F = (\pi_{X}^{*}E) \otimes (\pi_{Y}^{*}F) \]
their \textit{external product}, where $\pi_{X}$ and$\pi_{Y}$ are the corresponding projections from the product $X \times Y$.
This induces a pairing
\[ K(X) \otimes K(Y) \to K(X \times Y). \]

\begin{rem}
  If we take $Y = X$, then we can recover the ring structure on $K(X)$ by pulling back along the diagonal $\Delta \colon X \to X \times X$, because $E \otimes F = \Delta^{*}(E \boxtimes F)$.
\end{rem}

Assume now that $X$ and $Y$ have basepoints.
If the class $\xi_{1} \in K(X)$ restricts to zero over the basepoint $x_{0} \in X$ and the class $\xi_{2} \in K(Y)$ restricts to zero over the basepoint $y_{0} \in Y$, then their external product $\xi_{1} \boxtimes \xi_{2} \in K(X \times Y)$ restricts to zero over $X \vee Y$.
In particular also over the basepoint $(x_{0},y_{0}) \in X \times Y$, so we have $\xi_{1} \boxtimes \xi_{2} \in \tilde{K}(X \times Y)$.
Since it restricts to zero over $X \vee Y$, identifying $\tilde{K}(X \vee Y) \cong \tilde{K}(X) \oplus \tilde{K}(Y)$, we deduce that $\xi_{1} \boxtimes \xi_{2}$ maps to zero in $\tilde{K}(X) \oplus \tilde{K}(Y)$ under the isomorphism from \cite[Corollary 2.4.8]{ati67} seen at the end of the previous subsection.
So we may regard $\xi_{1} \boxtimes \xi_{2}$ as an element in $\tilde{K}(X \wedge Y)$.
Therefore the external product defines a product
\[ \tilde{K}(X) \otimes \tilde{K}(Y) \to \tilde{K}(X \wedge Y). \]
Following \cite[p.~54]{hat03}, we can summarize all these identifications and splittings into the following commutative diagram:

\begin{center}
  \begin{tikzcd}
    K(X) \otimes K(Y) \arrow{d} &[-12mm] \cong &[-12mm] (\tilde{K}(X) \otimes \tilde{K}(Y)) \arrow{d} &[-12mm] \oplus &[-12mm] \tilde{K}(X) \arrow[equal]{d} &[-12mm] \oplus &[-12mm] \tilde{K}(Y) \arrow[equal]{d} &[-12mm]\oplus &[-12mm] \mathbb{Z} \arrow[equal]{d} \\
    K(X \times Y) & \cong & \tilde{K}(X \wedge Y) & \oplus & \tilde{K}(X) & \oplus & \tilde{K}(Y) & \oplus & \mathbb{Z}
  \end{tikzcd}
\end{center}

From this, using associativity of smash products on compact Hausdorff spaces\footnote{Or more generally on compactly generated spaces.}, we deduce the existence of pairings

\[ \tilde{K}^{-n}(X) \otimes \tilde{K}^{-m}(Y) \to \tilde{K}^{-n-m}(X \wedge Y) \]

for all $n, m \in \mathbb{N}$.
And if $A \subseteq X$ and $B \subseteq Y$ are closed subspaces, then we may consider the corresponding pairing for the pointed spaces $X/A$ and $Y/B$, which yields pairings

\[ K^{-n}(X,A) \otimes K^{-m}(Y,B) \to K(X \times Y, (X \times B) \cup (A \times Y)) \]

for all $n, m \in \mathbb{N}$, where we are using the natural identification $(X \times Y)/((X \times B)\cup (A \times Y)) = X/A \wedge X/B$.

In particular, taking $Y = X$ and $B = A = \varnothing$, this defines a graded-commutative ring structure on

\[ K^{\#}(X) := \bigoplus_{n \in \mathbb{N}} K^{-n}(X), \]

and taking only $A = \varnothing$ we obtain a graded $K^{\#}(X)$-module structure on

\[ K^{\#}(X,B) := \bigoplus_{n \in \mathbb{N}} K^{-n}(X,B). \]

We use the periodicity isomorphism \cite[Theorem 2.4.9]{ati67} to identify $K^{-n}$ with $K^{-n-2}$ and to extend the definition of the various functors $K^{n}$ to all $n \in \mathbb{Z}$.
Because of this periodicity we only care about $K^{0}$ and $K^{1}$.
So we define:

\begin{defn}
  Let $X$ be a compact Hausdorff space.
  \begin{enumerate}
    \item We define $K^{*}(X) = K^{0}(X) \oplus K^{1}(X)$.
    \item If $x_{0} \in X$ is a baspoint, then we define $\tilde{K}^{*}(X) = \tilde{K}^{0}(X) \oplus \tilde{K}^{1}(X)$.
    \item If $A \subseteq X$ is a closed subset, then we define $K^{*}(X,A) = K^{0}(X,A) \oplus K^{1}(X,A)$.
  \end{enumerate}
  In particular, we still have
  \[ K^{*}(X) = K^{*}(X, \varnothing) = \tilde{K}^{*}(X^{+}). \]
\end{defn}

The previously discussed ring and module structures on $K^{\#}(X)$ and $K^{\#}(X,A)$ yield a $\mathbb{Z}/2\mathbb{Z}$-graded ring structure on $K^{*}(X)$ and a $\mathbb{Z}/2\mathbb{Z}$-graded $K^{*}(X)$-module strucutre on $K^{*}(X,A)$.

\subsection{The six-term exact sequence}

Let $X$ be a compact Hausdorff space and let $A \subseteq X$ be a closed subspace.
Using the periodicity isomorphism \cite[Theorem 2.4.9]{ati67}, we encoded the long exact sequence of the pair $(X,A)$ into a \textit{six-term exact sequence}

\begin{center}
  \begin{tikzcd}
    K^{0}(X,A) \arrow{r} & K^{0}(X) \arrow{r} & K^{0}(A) \arrow{d} \\
    K^{1}(A) \arrow{u} & K^{1}(X) \arrow{l} & K^{1}(X,A) \arrow{l}
  \end{tikzcd}
\end{center}

With the notation of the previous subsection, after checking that the coboundary map $K^{-1}(A) \to K^{0}(X,A)$ is a $K(X)$-module morphism \cite[Lemma 2.6.0]{ati67}, we can rewrite this six-term exact sequence into the \textit{exact triangle}

\begin{center}
  \begin{tikzcd}
    K^{*}(X) \arrow{rr} & & K^{*}(A) \arrow{dl} \\
    & K^{*}(X,A) \arrow{ul} & 
  \end{tikzcd}
\end{center}

of $K^{*}(X)$-module morphisms.

\section{Thom spaces}

\begin{defn}[Thom space]
  Let $p \colon E \to X$ be a complex vector bundle over a compact Hausdorff space $X$.
  Then we define its \textit{Thom space} $X^{E}$ as the one-point compactification of $E$.
\end{defn}

The goal of this talk is to study the $K$-theory of $X^{E}$.
Recall that all our base spaces are always assumed to be compact Hausdorff spaces; this explains why we want to study the one-point compactification $X^{E}$ rather than $E$ itself.
Another reason is that $E$ is homotopy equivalent to $X$, so at the level of cohomological invariants we would not obtain any new interesting information.

Whenever we need to think of $X^{E}$ concretely, we will use one of two alternative descriptions.
After fixing a metric on $E$, we can think of $X^{E}$ as the quotient space $B(E)/S(E)$, where $B(E)$ is the disc bundle associated to $E$ and $S(E)$ is the sphere bundle associated to $E$.
Explicitly, $B(E)$ consists of all vectors in $E$ with length at most $1$, and $S(E)$ consists of all vectors in $E$ with length exactly $1$.
Alternatively, we may construct $X^{E}$ as the quotient $P(E \oplus 1) / P(E)$, where by $1$ we mean the trivial line bundle $X \times \mathbb{C}$ and where we regard $P(E) \subseteq P(E \oplus 1)$ as the image of the section given by
\begin{align*}
  P(E) & \to P(E \oplus 1) \\
  [(v_{1}, \ldots, v_{n})] & \mapsto [(v_{1}, \ldots, v_{n}, 0)]
\end{align*}

Thus we are adding a ``common point at infinity'' to all the fibers of $E \to X$.

Quoting Atiyah, ``We shall now digress for some time to give an alternative and often illuminating description of $K(X,A)$ which has particular relevance for products.''
This alternative description will also allow us to find a canonical element in the $K$-theory of the Thom space $X^{E}$.
This element will play a central role in the Thom isomorphism theorem.

This detour is indeed rather lengthy, occupying most of \cite[\S 2.6]{ati67}.
The upshot is the following.
A \textit{complex of vector bundles} on $X$ consists of a sequence
\[ 0 \to E_{n} \xrightarrow{\sigma_{n}} E_{n-1} \xrightarrow{\sigma_{n-1}} \cdots \to E_{0} \to 0 \]
of vector bundles on $X$ such that $\sigma_{i} \circ \sigma_{i+1} = 0$ for all $i \in \{1, \dots, n-1\}$.
A complex of vector bundles $E_{\bullet}$ exact over the subspace $A$ defines an element $\chi (E_{\bullet}) \in K(X,A)$.
If $A = \varnothing$, this element is given by the formula

\[ \chi(E_{\bullet}) = \sum_{i = 0}^{n} (-1)^{i}[E_{i}] \in K(X). \]

If $Y$ is another compact Hausdorff space, $B \subseteq Y$ is a closed subspace and $F_{\bullet}$ is a complex of vector bundles on $Y$ exact over $B$, then the external tensor product defines a complex of vector bundles $E_{\bullet} \boxtimes F_{\bullet}$ on $X \times Y$ which is exact over $X \times B \cup A \times Y$.
We have then

\[ \chi( E_{\bullet} \boxtimes F_{\bullet}) = \chi(E_{\bullet}) \chi(F_{\bullet}), \]

yielding a new and convenient way to understand the relative external product

\[ K(X,A) \times K(Y,B) \to K(X \times Y, X \times B \cup A \times Y). \]

\begin{exmp}
  We have seen that $\tilde{K}^{2}(S^{2}) \cong \mathbb{Z}$ as an abelian group, generated by $[H] - 1 \in K(S^{2})$.
  The line bundle $H \to S^{2}$ is the hyperplane bundle obtained from regarding $S^{2}$ as the projective bundle $P(\{ * \} \times \mathbb{C}^{2})$ over $\{ * \}$.
  We again change our point of view slightly and regard $S^{2}$ as the quotient $B(\mathbb{C})/S(\mathbb{C})$, where $B(-)$ and $S(-)$ denote here the unit disc and unit sphere.
  We consider the complex of vector bundles on $B(\mathbb{C})$ given by
  \[ 0 \to B(\mathbb{C}) \times \Lambda^{0}\mathbb{C} \xrightarrow{\alpha} B(\mathbb{C}) \times \Lambda^{1}\mathbb{C} \to 0, \]
  where $\alpha(v,w) := (v,v \wedge w)$, i.e.~$\alpha(v, \lambda) = (v, v \lambda)$ under the identifications $\Lambda^{0}\mathbb{C} = \mathbb{C}$ and $\Lambda^{1}\mathbb{C} = \mathbb{C}$.
  Let us call $E_{1} := B(\mathbb{C}) \times \Lambda^{0}\mathbb{C}$ and $E_{0} := B(\mathbb{C}) \times \Lambda^{1} \mathbb{C}$.
  In this case both $E_{1}$ and $E_{0}$ are the trivial line bundle, and the differential in the complex is an isomorphism over all points $v \in B(\mathbb{C}) \setminus \{0 \}$.
  In particular, the complex $E_{\bullet}$ is exact over $S(\mathbb{C})$.
  Hence, this complex defines an element
  \[ \chi(E_{\bullet}) \in K(B(\mathbb{C}),S(\mathbb{C})) = \tilde{K}(S^{2}). \]

  Let us go through the proof of \cite[Lemma 2.6.7]{ati67} with this example and see what $\chi(E_{\bullet})$ is.
  Let $X_{0}$ and $X_{1}$ be two copies of $B(\mathbb{C})$, let $A = S(\mathbb{C})$ and let $Y = X_{0} \cup_{A} X_{1}$ be the result of gluing the two disjoint copies of $B(\mathbb{C})$ along $S(\mathbb{C})$.
  Using $\alpha$ as a clutching function we obtain a vector bundle $[E_{1}, \alpha, E_{0}] \in K(Y) = K(S^{2})$.
  The clutching function defined by $\alpha$ is given by $S^{1} \ni z \mapsto (\lambda \mapsto z \lambda)$.
  Let $r_{1} \colon Y \to X_{1}$ be the retraction of $X_{1}$ which is given by sending a point on $X_{0}$ to the corresponding point on $X_{1}$, and let $i_{1} \colon X_{1} \to Y$ denote the inclusion into the pushout.
  This defines a splitting of the long exact sequence from \cite[Proposition 2.4.4]{ati67}, hence an isomorphism
  
  \begin{align*}
    K(Y) & \cong K(Y,X_{1}) \oplus K(X_{1}) \\
    \xi & \mapsto (\xi - r_{1}^{*}i_{1}^{*}\xi,i_{1}^{*}\xi),
  \end{align*}
  where by the first component $\xi - r_{1}^{*}i_{1}^{*}\xi$ we really mean the bundle over $X/A = Y/X_{1}$ whose pullback along the quotient morphism $q \colon Y \to Y/X_{1}$ is that element of $K(Y)$.
  Since $q$ is the quotient of a contractible subspace, it induces a bijection on isomorphism classes of vector bundles \cite[Lemma 1.4.8]{ati67}.
  Thus we regard $\chi(E_{\bullet}) = [E_{1}, \alpha, E_{0}] - r_{1}^{*}i_{1}^{*}[E_{1}, \alpha, E_{0}] \in K(X_{0},A)$.
  Since $[E_{1}, \alpha, E_{0}]$ has clutching function $z \mapsto (\lambda \mapsto z\lambda)$, we have $[E_{1}, \alpha, E_{0}] = [H]^{-1}$ \cite[p.~49]{ati67}.
  From the equation $([H] - 1)^{2} = 0$ in $K(S^{2})$ we deduce that $[H]^{-1} = 2 - [H]$.
  And since $X_{1}$ is contractible, we have $r_{1}^{*}i_{1}^{*}[E_{1}, \alpha, E_{0}] = 1$.
  Therefore $\chi(E_{\bullet}) = 1 - [H]$.
\end{exmp}

More generally, let $V$ be an $n$-dimensional complex vector space and let $v \in V$ be a vector.
For each $i \in \mathbb{N}$ we consider the morphism

\begin{align*}
  \Lambda^{i}V & \to \Lambda^{i+1}V \\
  \alpha & \mapsto v \wedge \alpha
\end{align*}

These morphisms define a complex $\Lambda^{\bullet}V$ which is exact if $v \neq 0$, see Theorem 8.11 Keith Conrad's expository notes on exterior powers, available at \url{https://kconrad.math.uconn.edu/blurbs/linmultialg/extmod.pdf}.
So considering the complex $B(V) \times \Lambda^{\bullet}V$ of vector bundles over the unit disc $B(V)$ whose differentials over $v \in B(V)$ are given by exterior product with $v$ as above, we obtain an element in $\lambda_{V} \in K(B(V),S(V)) = \tilde{K}(S^{2n})$.
From our new description of the external product we see that

\[ \lambda_{V} = (-1)^{n}([H]-1)^{\boxtimes n} \in \tilde{K}(S^{2n}), \]

because $\Lambda^{\bullet}(V \oplus W) = (\Lambda^{\bullet}V) \otimes (\Lambda^{\bullet}W)$.
This $\lambda_{V}$ is the canonical generator of $\tilde{K}(S^{2n})$ up to a sign \cite[Corollary 2.12]{hat03}.

Globalizing the previous discussion, if $p \colon E \to X$ is a vector bundle of rank $n$ over a compact Hausdorff space $X$, then we consider the associated unit disc bundle $B(E)$ and the associated unit sphere bundle $S(E)$.
Then $\Lambda^{\bullet}(p^{*}E)$ is a complex of vector bundles over $B(E)$, with the differential over the point $v \in B(E) \subseteq E$ corresponding to the exterior product by $v$ as above.
This complex is then exact over $S(E)$, so we obtain an element $\lambda_{E} \in K(B(E),S(E)) = \tilde{K}(X^{E})$ which we call the \textit{Thom class} associated to $E$.
From the above discussion and from the construction of this complex we deduce the following two properties:

\begin{enumerate}[label=(\Alph*)]
  \item The Thom class $\lambda_{E}$ restricts to a generator of $\tilde{K}(\{ x \}^{E}) \cong \tilde{K}(S^{2n})$ for each $x \in X$.
  \item Under the appropriate identifications of disc and sphere bundles, $\lambda_{(E \oplus F)} = \lambda_{E} \boxtimes \lambda_{F}$ in $\tilde{K}(X^{E \oplus F})$.
\end{enumerate}

We turn now to our second description of $X^{E}$.
Let us consider the hyperplane bundle $H$ over $P(E \oplus 1)$.
Tensoring the inclusion $H^{*} \subseteq \pi^{*}(E \oplus 1)$ with $H$ we obtain morphisms
\[ 1 \cong H \otimes H^{*} \hookrightarrow H \otimes \pi^{*}(E \oplus 1) = (H \otimes \pi^{*}(E)) \oplus H, \]
where $\pi \colon P(E \oplus 1) \to X$ denotes the projection.
Composing further with the projection onto the first factor we obtain a natural section $s \in \Gamma(H \otimes \pi^{*}(E))$.
We use this section to define a complex of vector bundles $\Lambda^{\bullet}(H \otimes \pi^{*}(E))$ over $P(E \oplus 1)$, with the differential over a point $[(v_{1}, \ldots, v_{n}, \lambda)]$ given by exterior product with the vector $s([(v_{1}, \ldots, v_{n}, \lambda)]))$.
The resulting complex is exact outside of the zero locus of the section $s$.
This zero locus is given by the points $[(v_{1}, \ldots, v_{n}, \lambda)] \in P(E \oplus 1)$ such that $(v_{1}, \ldots, v_{n}) = 0$, hence it is the isomorphic image of $X$ under the section $X \to P(E \oplus 1)$ given by $x \mapsto [(0, \ldots, 0, 1)]$.
So the complex is exact over all points in the image of the morphism $P(E) \hookrightarrow P(E \oplus 1)$ given by $[(v_{1}, \ldots, v_{n})] \mapsto [(v_{1}, \ldots, v_{n}, 0)]$.
Therefore it defines an element
\[ \chi(\Lambda^{\bullet}(H \otimes \pi^{*}(E))) \in K(P(E \oplus 1), P(E)) = \tilde{K}(X^{E}). \]
It follows from the definition of $\chi$ \cite[Definition 2.6.2]{ati67} that the image of this element in $K(P(E \oplus 1))$ is givey by
\[ \sum_{i = 1}^{n} (-1)^{i} [H]^{i}[\Lambda^{i}E]. \]
We claim now that this element is in fact $\lambda_{E}$ under the identification $K(P(E \oplus 1), P(E)) = K(B(E),S(E))$.
Indeed, we look at the complement of $P(E)$ inside $P(E \oplus 1)$, with $P(E)$ embedded in $P(E \oplus 1)$ as above.
This complement consists of points $[(v_{1}, \ldots, v_{n}, \lambda)]$ such that $\lambda \neq 0$.
We can rescale to obtain a representative of the form $[(v_{1}/\lambda, \ldots, v_{n}/\lambda,1)]$ which is then unique.
This shows that $P(E \oplus 1) \setminus P(E) \cong E$.
Moreover, when we restrict $H$ to this copy of $E$ inside $P(E \oplus 1)$, we obtain the trivial line bundle.
Indeed, it suffices to show that $H^{*}|_{E}$ is the trivial line bundle, and the section $[(v_{1}, \ldots, v_{n}, 1)] \mapsto (v_{1}, \ldots, v_{n},1)$ does not vanish on any point of $E \subseteq P(E \oplus 1)$.
So $H|_{E}$ is the trivial line bundle and the complex of vector bundles $\Lambda^{\bullet}(H \otimes \pi^{*}(E))$ gets identified with the complex of vector bundles $\Lambda^{\bullet}(p^{*}E)$ over $B(E)$ under the embedding of $B(E) \subseteq E$ in $P(E \oplus 1)$ described above, at least at the level of vector bundles appearing on the complex on each degree.
Moreover, the section $s$ evaluated at a point $[(v_{1}, \ldots, v_{n},1)] \in B(E) \subseteq P(E \oplus 1)$ corresponds precisely to the vector $(v_{1}, \ldots, v_{n}) \in E$ itself, so the differentials of the two complexes are also identified.
Hence $\chi(\Lambda^{\bullet}(H \otimes \pi^{*}(E)))$ corresponds to $\lambda_{E}$ under the identification $K(P(E \oplus 1), P(E)) = K(B(E),S(E))$, as we wanted to show.

\section{Proof: the case of decomposable vector bundles}

In Vera's talk we have seen:

\begin{prop}\label{prop:sumnongraded}
  Let $X$ be a compact Hausdorff space.
  Let $L_{1}, \ldots, L_{n}$ be line bundles over $X$ and let $H$ be the hyperplane bundle over $P(L_{1} \oplus \ldots \oplus L_{n})$.
  Then the $K(X)$-algebra morphism sending $t \mapsto [H]$ induces a $K(X)$-algebra isomorphism
  \[ K(X)[t]/\prod_{i=1}^{n}([L_{i}]t - 1) \to K(P(L_{1} \oplus \ldots \oplus L_{n})). \]
\end{prop}

We wish to combine this result with the following:

\begin{lm}\label{lm:gradedvsold}
  Let $X$ be a compact Hausdorff space.
  Then there is a canonical group isomorphism
  \[ K^{*}X \cong K(X \times S^{1}). \]
  \begin{proof}
    We follow the argument in \cite[p.~38]{wir12}.
    If $X = \varnothing$, then both sides are just $K(\varnothing) = 0$.
    Otherwise, we pick a basepoint $x_{0} \in X$.
    This induces a decomposition $\tilde{K}(X \times S^{1}) \cong \tilde{K}(S^{1} \wedge X) \oplus \tilde{K}(S^{1}) \oplus \tilde{K}(X)$ \cite[Corollary 2.4.8]{ati67}.
    But $\tilde{K}(S^{1}) = 0$ and $\tilde{K}(S^{1} \wedge X) = \tilde{K}^{1}(X) = K^{1}(X)$, so we have
    \[ \tilde{K}(S^{1} \times X) \cong K^{1}(X) \oplus \tilde{K}(X). \]
    Therefore we obtain the desired isomorphism after adding the $K^{0}$ of the basepoint of $S^{1} \times X$ and of $X$ respectively on both sides.
  \end{proof}
\end{lm}

\begin{lm}\label{lm:gradedvsoldbundle}
  In the situation of \Cref{lm:gradedvsold}, let $\pi \colon X \times S^{1} \to X$ be the projection and let $p \colon E \to X$ be a vector bundle on $X$.
  Then $P(\pi^{*}E) = P(E) \times S^{1}$.
  In particular, by \Cref{lm:gradedvsold}, there is a canonical group isomorphism
  \[ K^{*}(P(E)) \cong K(P(\pi^{*}E)). \]
  \begin{proof}
    We need to show that there exists a canonical homeomorphism $P(\pi^{*}E) \cong P(E) \times S^{1}$.
    So let $[(v_{1}, \ldots, v_{n})] \in P(\pi^{*}E)$ be a point over $(p(v_{1}, \ldots, v_{n}), z) \in X \times S^{1}$.
    We send this point to $([(v_{1}, \ldots, v_{n})], z) \in P(E) \times S^{1}$.
    This is a continuous bijection between compact Hausdorff spaces, hence a homeomorphism.
  \end{proof}
\end{lm}

Now we can deduce the following:

\begin{prop}\label{prop:sumgraded}
  Let $X$ be a compact Hausdorff space.
  Let $L_{1}, \ldots, L_{n}$ be line bundles over $X$ and let $H$ be the hyperplane bundle over $P(L_{1} \oplus \ldots \oplus L_{n})$.
  Then the $K^{*}(X)$-algebra morphism sending $t \mapsto [H]$ induces a $K^{*}(X)$-algebra isomorphism
  \[ K^{*}(X)[t]/\prod_{i=1}^{n}([L_{i}]t - 1) \to K^{*}(P(L_{1} \oplus \ldots \oplus L_{n})). \]
  \begin{proof}
    It suffices to show bijectivity, so it suffices to show that the underlying group morphism is an isomorphism.
    Under the isomorphism $K^{*}(X) \cong K(X \times S^{1})$ from \Cref{lm:gradedvsold}, the line bundle $L_{i}$ corresponds to $\pi^{*}L_{i}$ for each $i \in \{1, \ldots, n\}$.
    Hence we have a commutative diagram of groups as follows:
    
    \begin{center}
      \begin{tikzcd}
        K^{*}(X)[t]/\prod_{i=1}^{n}([L_{i}]t-1) \arrow{d} \arrow{r} & K^{*}(P(L_{1} \oplus \ldots \oplus L_{n})) \arrow{d} \\
        K(X \times S^{1})/\prod_{i=1}^{n}([\pi^{*}L_{i}]t - 1) \arrow{r} & K(P(\pi^{*}L_{1} \oplus \ldots \oplus \pi^{*}L_{n}))
      \end{tikzcd}
    \end{center}
    
    The two vertical arrows are group isomorphisms by \Cref{lm:gradedvsold} and \Cref{lm:gradedvsoldbundle}, and the bottom horizontal arrow is the isomorphism from \Cref{prop:sumnongraded}.
    Hence the top horizontal arrow is an isomorphism, which is what we needed to show.
  \end{proof}
\end{prop}

\begin{rem}\label{rem:free}
  In the situation of \Cref{prop:sumgraded}, the elements $1, [H], [H]^{2}, \ldots, [H]^{n-1}$ form a basis of $K^{*}(P(E))$ over $K^{*}(X)$, because each coefficient $[L_{i}]$ is an invertible element in $K^{*}(X)$.
  Let us denote by $M^{*}$ the free abelian group generated by $1, [H], [H]^{2}, \ldots, [H]^{n-1}$ endowed with the $\mathbb{Z}/2\mathbb{Z}$-grading induced by the degree of each $[H]^{i}$, which is zero for all $[H]^{i}$ anyway.
  Then the natural map
  \[ K^{*}(X) \otimes M^{*} \to K^{*}(P(E)) \]
  is an isomorphism compatible with the $\mathbb{Z}/2\mathbb{Z}$-grading.
\end{rem}

We are now ready to prove the Thom isomorphism theorem in the case of decomposable vector bundles:

\begin{prop}\label{prop:sumthom}
  Let $X$ be a compact Hausdorff space.
  Let $L_{1}, \ldots, L_{n}$ be line bundles over $X$, let $E := L_{1} \oplus \ldots \oplus L_{n}$ be their direct sum and let $\lambda_{E} \in \tilde{K}(X^{E})$ be the Thom class.
  Then $\tilde{K}^{*}(X^{E})$ is the free $K^{*}(X)$-module generated by $\lambda_{E}$.
  \begin{proof}
    We think of $X^{E}$ as $P(E \oplus 1)/P(E)$.
    We have seen that in this concrete incarnation, the image of $\lambda_{E}$ in $K(P(E\oplus 1))$ is
    \[ \sum_{i=1}^{n} (-1)^{i}[H]^{i}[\Lambda^{i}E] = \prod_{i=1}^{n}(1 - [L_{i}][H]), \]
    where the equality comes from applying the canonical isomorphisms $\Lambda^{m}(M\oplus N) \cong \oplus_{p+q = m} (\Lambda^{p}M )\otimes (\Lambda^{q}N)$ iteratively, cf.~\cite[p.~42]{wir12}.
    We apply \Cref{prop:sumgraded} to $E$ and to $E \oplus 1$.
    Since the hyperplane bundle on $P(E \oplus 1)$ restricts to the hyperplane bundle on $P(E)$, the long exact sequence of $K^{*}(X)$-modules associated to the pair $(P(E \oplus 1), P(E))$ looks as follows:
    \[ \cdots \to \tilde{K}^{*}(X^{E}) \to K^{*}(X)[t]/(f) \xrightarrow{t \mapsto s} K^{*}(X)[s]/(g) \to \cdots, \]
    with $g(s) = \prod_{i=1}^{n}([L_{i}]s -1)$ and $f(t) = (t-1)g(t)$.
    In particular, the morphism induced by $t \mapsto s$ is surjective, so we obtain a short exact sequence of $K^{*}(X)$-modules as follows:
    \[ 0 \to \tilde{K}^{*}(X^{E}) \to K^{*}(X)[t]/(f) \to K^{*}(X)[s]/(g) \to 0. \]
    As mentioned in the beginning of the proof, we have
    \[ \lambda_{E} \mapsto (-1)^{n}g(t), \]
    and this element generates the kernel of the morphism induced by $t \mapsto s$ as an ideal in $K^{*}(X)[t]/(f)$.
    Division by monic polynomials or by polynomials with a unit as leading coefficient still works with non-commutative coefficient rings, so every element in the kernel of the morphism induced by $t \mapsto s$ can be written as $\xi g(t)$ for some $\xi \in K^{*}(X)$.
    Thus $\tilde{K}(X^{E})$ is the free $K^{*}(X)$-module generated by $\lambda_{E}$.
  \end{proof}
\end{prop}

\section{Proof: the general case}

In order to deduce the general case from the case of decomposable vector bundles, we need the following:

\begin{lm}\label{lm:lerayhirsch}
  Let $\pi \colon B \to X$ be a morphism of compact Hausdorff spaces.
  Let $\mu_{1}, \ldots, \mu_{n}$ be homogeneous elements of $K^{*}(B)$ and let $M^{*}$ be the free abelian group generated by $\mu_{1}, \ldots, \mu_{n}$ endowed with the $\mathbb{Z}/2\mathbb{Z}$-grading induced by the degree of each $\mu_{i}$.
  Suppose that each $x \in X$ has a neighborhood $U$ such that for all compact $A \subseteq U$ the natural map
  \[ K^{*}(A) \otimes M^{*} \to K^{*}(\pi^{-1}(A)) \]
  is an isomorphism.
  Then, for any $A \subseteq X$, the natural map
  \[ K^{*}(X,A) \otimes M^{*} \to K^{*}(B, \pi^{-1}(A)) \]
  is an isomorphism.
  
  \begin{proof}
    Let us call a subset $U \subseteq X$ \textit{good} if it has the property that for all compact $A \subseteq U$ the natural map
    \[ K^{*}(A) \otimes M^{*} \to K^{*}(\pi^{-1}(A)) \]
    is an isomorphism.
    Suppose $U$ is good and compact.
    Then consider the long exact sequence of the pair $(U,A)$:
    \[ \cdots \to K^{*}(U,A) \to K^{*}(U) \to K^{*}(A) \to \cdots \]
    Since $M^{*}$ is flat, we obtain a new long exact sequence
    \[ \cdots \to K^{*}(U,A) \otimes M^{*} \to K^{*}(U) \otimes M^{*} \to K^{*}(A) \otimes M^{*} \to \cdots \]
    Using the assumption that $U$ is good we obtain the long exact sequence
    \[ \cdots \to K^{*}(U, A) \otimes M^{*} \to K^{*}(\pi^{-1}(U)) \to K^{*}(\pi^{-1}(A)) \to \cdots \]
    Now we apply the $5$-lemma to this sequence and to the long exact sequence of the pair $(\pi^{-1}(U), \pi^{-1}(A))$ with the identities and the natural map $K^{*}(U,A) \otimes M^{*} \to K^{*}(\pi^{-1}(U), \pi^{-1}(A))$ as vertical arrows.
    This shows that the natural map $K^{*}(U,A) \otimes M^{*} \to K^{*}(\pi^{-1}(U), \pi^{-1}(A))$ is an isomorphism as well.
    So in order to prove the lemma, it suffices to show that $X$ is a good subset.

    By assumption we can cover $X$ by good open subsets.
    Since $X$ is compact we may find a finite subcover, so that $X$ is covered by finitely many good open subsets.
    If we show the the union of two good open subsets is a good open subset, then we are done.

    So let $U_{1}$ and $U_{2}$ be good open subsets of $X$.
    Let $A \subseteq U_{1} \cup U_{2}$ be a compact subspace.
    Then we can write $A = A_{1} \cup A_{2}$ with $A_{1} \subseteq U_{1}$ and $A_{2} \subseteq U_{2}$ and with $A_{1}$ and $A_{2}$ still compact.
    Indeed, $X$ is compact Hausdorff, hence normal.
    Thus we can find disjoint open subsets $V$ and $W$ such that $A \cap (X \setminus U_{2}) \subseteq V$ and $(X \setminus U_{1}) \subseteq W$.
    Then we may take $A_{1} := A \cap (X \setminus W)$ and $A_{2} := A \cap (X \setminus V)$.
    We have $A/A_{2} = A_{1}/(A_{1} \cap A_{2})$ and $\pi^{-1}(A_{1})/\pi^{-1}(A_{1} \cap A_{2}) = \pi^{-1}(A_{1})/(\pi^{-1}(A_{1}) \cap \pi^{-1}(A_{2})) = \pi^{-1}(A)/\pi^{-1}(A_{2})$, so $K^{*}(A_{1}, A_{1} \cap A_{2}) \otimes M^{*} \to K^{*}(\pi^{-1}(A_{1}), \pi^{-1}(A_{1} \cap A_{2}))$ being an isomorphism implies that $K^{*}(A, A_{2}) \otimes M^{*} \to K^{*}(\pi^{-1}(A), \pi^{-1}(A_{2}))$ is an isomorphism.
    Combining this with the assumption that $U_{2}$ is good and applying the $5$-lemma to the long exact sequences of the corresponding pairs we deduce that $K^{*}(A) \otimes M^{*} \to K^{*}(\pi^{-1}(A))$ is an isomorphism, thus showing that $U_{1} \cup U_{2}$ is good and finishing the proof.
  \end{proof}
\end{lm}

\begin{cor}\label{cor:generalgraded}
  Let $X$ be a compact Hausdorff space.
  Let $E$ be a vector bundle of rank $n$ over $X$ and let $H$ be the hyperplane bundle over $P(E)$.
  Then $K^{*}(P(E))$ is a free $K^{*}(X)$-module generated by $1, [H], [H]^{2}, \ldots, [H]^{n-1}$, and $[H]$ satisfies the equation
  \[ \sum_{i = 0}^{n} (-1)^{i}[H]^{i}[\Lambda^{i}E] = 0. \]
  \begin{proof}
    We wish to apply \Cref{lm:lerayhirsch} to the morphism $\pi \colon P(E) \to X$, the homogeneous elements $1, [H], [H]^{2}, \ldots, [H]^{n-1}$ in $K^{*}(P(E))$ and the subspace $A = \varnothing$.
    Let $x \in X$ be a point and let $U \subseteq X$ be an open neighborhood of $x$ over which $E$ is trivial, hence a direct sum of $n$ trivial line bundles.
    The restriction of $E$ to any compact subspace of $U$ is then again the direct sum of $n$ trivial line bundles.
    Therefore \Cref{prop:sumgraded} implies that $U$ is good, cf.~also \Cref{rem:free}.
    Since $K^{*}(-,\varnothing) = K^{*}(-)$, \Cref{lm:lerayhirsch} finishes the proof.
  \end{proof}
\end{cor}

And from this we can finally deduce the following

\begin{cor}[Thom isomorphism theorem]\label{cor:generalthom}
  Let $X$ be a compact Hausdorff space.
  Let $E$ be a vector bundle of rank $n$ over $X$ and let $\lambda_{E} \in \tilde{K}(X^{E})$ be the Thom class.
  Then $\tilde{K}^{*}(X^{E})$ is the free $K^{*}(X)$-module generated by $\lambda_{E}$.
  \begin{proof}
    We can use the same argument as in the proof of \Cref{prop:sumthom} but applying \Cref{cor:generalgraded} instead of \Cref{prop:sumgraded}.
  \end{proof}
\end{cor}

\bibliographystyle{alpha}
\bibliography{main}
\vfill

\end{document}
