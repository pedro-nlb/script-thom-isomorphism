\documentclass[12pt,a4paper]{amsart}

\usepackage[T1]{fontenc}
\usepackage[utf8]{inputenc}
\usepackage[british]{babel}
\usepackage{mathtools}
\usepackage{amsthm}
\usepackage{amssymb}
\usepackage{mathrsfs}
\usepackage{enumitem}
\usepackage{tikz-cd}
\usetikzlibrary{decorations.markings}
\usepackage{float}
\usepackage{hyperref}
\urlstyle{same}
\usepackage[noabbrev]{cleveref}

\theoremstyle{plain}
\newtheorem{thm}{Theorem}
\newtheorem*{thm*}{Theorem}
\newtheorem{lm}[thm]{Lemma}
\newtheorem{prop}[thm]{Proposition}
\newtheorem{cor}[thm]{Corollary}
\theoremstyle{definition}
\newtheorem{defn}[thm]{Definition}
\newtheorem{exmp}[thm]{Example}
\newtheorem{xca}[thm]{Exercise}
\theoremstyle{remark}
\newtheorem{rem}[thm]{Remark}
\Crefname{thm}{Theorem}{Theorems}
\Crefname{lm}{Lemma}{Lemmas}
\Crefname{prop}{Proposition}{Propositions}
\Crefname{cor}{Corollary}{Corollaries}
\Crefname{defn}{Definition}{Definitions}
\Crefname{exmp}{Example}{Examples}
\Crefname{xca}{Exercise}{Exercises}
\Crefname{rem}{Remark}{Remarks}

\title[The Thom isomorphism]{The Thom isomorphism}
\author[Pedro N\'{u}\~{n}ez]{Pedro N\'{u}\~{n}ez}
\address{Pedro N\'{u}\~{n}ez \newline
\indent Albert-Ludwigs-Universit\"{a}t Freiburg, Mathematisches Institut \newline
\indent Ernst-Zermelo-Straße 1, 79104 Freiburg im Breisgau (Germany)}
\email{\normalfont\href{mailto:pedro.nunez@math.uni-freiburg.de}{pedro.nunez@math.uni-freiburg.de}}
\renewcommand*{\urladdrname}{\itshape Homepage}
\urladdr{\normalfont\href{https://home.mathematik.uni-freiburg.de/nunez/}{https://home.mathematik.uni-freiburg.de/nunez}}
\thanks{The author gratefully acknowledges support by the DFG-Graduiertenkolleg GK1821 ``Cohomological Methods in Geometry'' at the University of Freiburg.}
\date{30th June 2021}

\setcounter{tocdepth}{1}
\sloppy
\makeatletter
\hypersetup{
  pdfauthor={\authors},
  pdftitle={\@title},
  colorlinks,
  linkcolor=[rgb]{0.2,0.2,0.6},
  citecolor=[rgb]{0.2,0.6,0.2},
  urlcolor=[rgb]{0.6,0.2,0.2}}
\makeatother

\begin{document}

\maketitle

\begin{abstract}
  Script for a talk of the Wednesday Seminar of the GK1821 at Freiburg during the Summer Semester 2021.
  The main reference is \cite[\S 2]{ati67}.
\end{abstract}

\tableofcontents

\begin{center}
  \textcolor{gray}{---parts in gray will be omitted during the talk---}
\end{center}

\section{Recollection from previous talks}

We start off with some recollections from \cite[\S 2]{ati67}.
We refer to \cite{ati67} and the appendix for previous conventions and other preliminaries.

\subsection{Notation and topological preliminaries}

For $n \in \mathbb{N}$ we denote by $S^{n}$ the $n$-dimensional sphere.
We think of it as the pointed compact Hausdorff space $I^{n}/\partial I^{n}$, where $I^{n} \subseteq \mathbb{R}^{n}$ is the unit interval and the basepoint is given by the equivalence class of any point in the boundary $\partial I^{n}$.
The unit interval $I^{1} = [0,1]$ is denoted simply by $I$.

Given pointed compact Hausdorff spaces $(X,x_{0})$ and $(Y,y_{0})$, we define their \textit{smash product} as
\[ X \wedge Y := X \times Y / X \vee Y, \]
where $X \vee Y := X \times \{y_{0}\} \cup \{ x_{0}\} \times Y$ is their \textit{wedge sum}.
The \textit{reduced suspension} of $(X,x_{0})$ is obtained from the usual suspension $(I\times X)/(\{0\}\times X \cup \{1\} \times X)$ by further collapsing the line $I \times \{ x_{0} \}$ joining the two vertices of the suspension along the basepoint.
Using our explicit description of $S^{1}$, we may rewrite the reduced suspension of $X$ as $S^{1} \wedge X$, and we denote it by $SX$.
The $n$-th iterated suspension is naturally homeomorphic to $S^{n} \wedge X$, and we denote it by $S^{n}X$.
Finally, the \textit{cone} of the pointed compact Hausdorff space $(X,x_{0})$ is defined as $CX := (I \times X )/ (\{0 \} \times X)$, and we take as basepoint the equivalence class of any point in the subspace that we are collapsing \cite[p.~68]{ati67}.

\subsection{The functor $K(-)$}
We follow \cite[p.~44]{ati67}.
Let $\mathbf{CHaus}$ be the category of compact Hausdorff topological spaces and let $\mathbf{Ring}$ be the category of commutative unital rings.
We have defined a functor $K \colon \mathbf{CHaus} \to \mathbf{Ring}$ which can be explicitly described as follows.
If $E \in \operatorname{Vect}(X)$ is a vector bundle on a compact Hausdorff space $X$, then we denote by $[E]$ its stable equivalence class, i.e.,

\[ [E] := \{ F \in \operatorname{Vect}(X) \mid \exists n \in \mathbb{N} \text{ s.t.~} E \oplus n \cong F \oplus n \}, \]

where we denote also by $n$ the trivial vector bundle of rank $n$ on $X$.
Then the underlying set of the ring $K(X)$ consists of formal differences $[E] - [F]$ for vector bundles $E, F \in \operatorname{Vect}(X)$.
The sum is given by
\[ ([E] - [F]) + ([E'] - [F']) = [E \oplus E'] - [F \oplus F'], \]
and the product is given by
\[ ([E] - [F]) ([E'] - [F']) = [(E \otimes E') \oplus (F \otimes F')] - [(E \otimes F') \oplus (E' \otimes F)]. \]
The element zero can be represented by $[0] - [0]$ and the element one by $[1] - [0]$, where again by $0$ we mean $X \times \{0\}$ and by $1$ we mean $X \times \mathbb{C}$.
Moreover, since $X$ is compact and Hausdorff, we can represent every element of $K(X)$ as $[E] - [n]$ for some $E \in \operatorname{Vect}(X)$ and some $n \in \mathbb{N}$.

If $f \colon X \to Y$ is a continuous map between compact Hausdorff topological spaces, then $K(f) =: f^{*}$ is given by

\begin{align*}
  f^{*} \colon K(Y) & \to K(X) \\
  [E] - [F] & \mapsto [f^{*}(E)] - [f^{*}(F)],
\end{align*}

and this ring homomorphism only depends on the homotopy class of $f$.

\subsection{The functor $\tilde{K}(-)$}
We follow \cite[p.~66]{ati67}.
Let $\mathbf{CHaus}^{+}$ be the category of pointed compact Hausdorff topological spaces and let $\mathbf{Rng}$ be the category of commutative non-unital rings.
We have defined a functor $\tilde{K} \colon \mathbf{CHaus}^{+} \to \mathbf{Rng}$ as follows.
Let $(X,x_{0})$ be pointed compact Hausdorff topological space and let $i \colon \{ x_{0} \} \to X$ denote the inclusion of the base point.
Then
\[ \tilde{K}(X) := \ker(i^{*}) \subseteq K(X). \]

\begin{rem}
  $\tilde{K}(X)$ is a non-unital subring: it is a subgroup closed under multiplication but it does not contain $1 \in K(X)$.
\end{rem}

If $c \colon X \to \{ x_{0} \}$ is the constant morphism to the basepoint, then $c^{*}$ induces a splitting $K(X) \cong \tilde{K}(X) \oplus K(x_{0})$ which is natural with respect to morphisms in $\mathbf{CHaus}^{+}$, hence $\tilde{K}(-)$ is a functor as claimed above.
If $f \colon (X,x_{0}) \to (Y,y_{0})$ is a morphism in $\mathbf{CHaus}^{+}$ given by a continuous function $f \colon X \to Y$ such that $f(x_{0}) = y_{0}$, then $\tilde{K}(f) =: f^{*}$ is induced by the restriction of $f^{*} \colon K(Y) \to K(X)$.

\begin{rem}
  We can recover the functor $K(-)$ from the functor $\tilde{K}(-)$ by adding disjoint basepoints:
  \[ K(X) \cong \tilde{K}(X^{+}), \]
  where $X^{+}$ is the pointed compact Hausdorff space obtained from adding a disjoint basepoint to the compact Hausdorff space $X$.
  This is a priori an isomorphism of non-unital rings, but we recover the unit $1 \in K(X)$ as the element corresponding to \color{red} finish this! \color{black}
  % TODO line above
\end{rem}

\subsection{The functor $K(-,-)$}
We follow \cite[p.~66]{ati67}.
Let $\mathbf{CHaus}^{2}$ be the category of pairs $(X,Y)$ consisting of a compact Hausdorff space $X$ and a compact Hausdorff subspace $Y \subseteq X$, or equivalently a compact Hausdorff space $X$ and a closed subspace $Y \subseteq X$.
We have defined a functor $K \colon \mathbf{CHaus}^{2} \to \mathbf{Rng}$ as follows.
For an object $(X,Y)$ in $\mathbf{CHaus}^{2}$, we set
\[ K(X,Y) := \tilde{K}(X/Y), \]

where the basepoint of $X/Y$ is the equivalence class of any point $y \in Y$, which we denote by $Y/Y$.
If $f \colon (X,Y) \to (Z,W)$ is a morphism in $\mathbf{CHaus}^{2}$ given by a continuous function $f \colon X \to Z$ such that $f(Y) \subseteq W$, then it induced a morphism $\bar{f} \colon (X/Y,Y/Y) \to (Z/W, W/W)$ in $\mathbf{CHaus}^{+}$.
Then $K(f) =: f^{*}$ is given by $\tilde{K}(\bar{f})$.

\subsection{The six-term exact sequence}

\begin{defn}[{\cite[Definition 2.4.1]{ati67}}]
  For $n \in \mathbb{N}_{>0}$ we define a functor $\tilde{K}^{-n} \colon \mathbf{CHaus}^{+} \to \mathbf{Rng}$ by setting
  \[ \tilde{K}^{-n}(X) := \tilde{K}(S^{n}X). \]
  Then we define a functor $K^{-n} \colon \mathbf{CHaus}^{2} \to \mathbf{Rng}$ via
  \[ K^{-n}(X,Y) := \tilde{K}^{-n}(X/Y), \]
  and finally a functor $K^{-n} \colon \mathbf{CHaus} \to \mathbf{Rng}$ with the formula
  \[ K^{-n}(X) := K^{-n}(X, \varnothing). \]
\end{defn}

\begin{rem}
  % TODO finish
  The last functor takes values in $\mathbf{Ring}$, the unit of $K^{-n}(X)$ being given by... \color{red} finish! \color{black}
\end{rem}

Recall from Vera's talk:

\begin{prop}[{\cite[Proposition 2.4.4]{ati67}}]
  For each pair $(X,Y)$ in $\mathbf{CHaus}^{2}$ there is a natural exact sequence
  \[ \cdots \to K^{-1}(Y) \xrightarrow{\delta} K^{0}(X,Y) \xrightarrow{j^{*}} K^{0}(X) \xrightarrow{i^{*}} K^{0}(Y), \]
  where $i \colon Y \to X$ and $j \colon (X,\varnothing) \to (X,Y)$ are the inclusions.
\end{prop}

\begin{rem}
  % TODO finish
  The morphism $\delta \colon K^{-1}(Y) \to K^{0}(X,Y)$ can be constructed as the composition $m^{*} \circ \theta^{-1}$, where $\theta$ is the isomorphism $K(X \cup CY)$... \color{red} finish! \color{black}
  But the important part is that $\delta$ is induced by a morphism of spaces, cf.~\cite[p.~77]{ati67}.
\end{rem}

Using the periodicity isomorphisms $\beta \colon K^{-n}(X) \to K^{-n-2}(X)$ from \cite[Theorem 2.4.9]{ati67} we extend the collection of functors $\{ K^{-n}(-,-) \}_{n \in \mathbb{N}}$ to a collection of funcotrs $\{ K^{n}(-,-) \}_{n \in \mathbb{Z}}$ inductively, i.e.~$K^{1} = K^{-1}$, $K^{2} = K^{0}$, $K^{3} =K^{1} = K^{-1}$, and so on.
Then we use the previously seen relations among the different $K$ functors to define collections of functors $\{ K^{n}(-) \}_{n \in \mathbb{Z}}$ and $\{ \tilde{K}^{n}(-) \}_{n \in \mathbb{Z}}$ as well.
As we saw at the end of Vera's talk, this allows us to extend the previous exact sequence to the right, and we rearrange this data into the following \textit{six-term exact sequence}

\begin{center}
  \begin{tikzcd}
    K^{0}(X,Y) \arrow{r} & K^{0}(X) \arrow{r} & K^{0}(Y) \arrow{d} \\
    K^{1}(Y) \arrow{u} & K^{1}(X) \arrow{l} & K^{1}(X,Y) \arrow{l}
  \end{tikzcd}
\end{center}

\begin{defn}
  Let $X$ be a compact Hausdorff space.
  Then we define
  \[ K^{*}(X) := K^{0}(X) \oplus K^{1}(X). \]
  Similarly, for a pair $(X,Y)$ in $\mathbf{CHaus}^{2}$ we define
  \[ K^{*}(X,Y) := K^{0}(X,Y) \oplus K^{1}(X,Y) \]
  and finally
  \[ \tilde{K}^{*}(X) := K^{*}(X,\varnothing). \]
\end{defn}

\begin{rem}
  % TODO finish
  Ring structure on $K^{*}$ \color{red} finish! \color{black}
  In fact, we get the triangle of \cite[p.~87]{ati67} in which all morphisms are $K^{*}(X)$-module morphisms.
\end{rem}

\section{Thom spaces and statement of the theorem}

Introduction and goal of the section: introduce Thom spaces and state the Thom isomorphism theorem.

Brief spoiler defining Thom spaces here already.

\subsection{Even dimensional spheres}

Recall the structure of the group $\tilde{K}^{0}(\mathbb{S}^{2n})$ determined in Vera's talk \cite[Corollary 2.12]{hat03}.
We already have a generator from this result in Hatcher's book.
We can describe this canonical generator---up to a sign---in terms of the exterior algebra on an $n$-dimensional $\mathbb{C}$-vector space $V$ \cite[p.~99]{ati67}.

\subsection{Thom spaces}

Generalize discussion on even dimensional spheres to Thom spaces as in \cite[p.~100]{ati67}.
Define the canonical $\lambda_{E} \in \tilde{K}(X^{E})$ and explain its properties.
Explain this also in terms of projectivizations of vector bundles plus a trivial line bundle, and recall Prof.~Huber's remark that this projectivization is a way to compactify the vector bundle.

\subsection{Statement of the theorem}

State the Thom isomorphism theorem \cite[Corollary 2.7.12]{ati67}.

\section{Proof for sums of line bundles}

In this section we work with a direct sum of line bundles

\[ E = L_{1} \oplus \ldots \oplus L_{m}. \]

\subsection{Recall $K^{0}(\mathbb{P}(E))$}

This is treated in \cite[Proposition 2.5.3]{ati67}; leave proof in gray, omitted during the talk.

\subsection{Computation of $K^{*}(\mathbb{P}(E))$}

This is \cite[Proposition 2.7.1]{ati67}.

\subsection{Thom isomorphism theorem in this case}

This is \cite[Proposition 2.7.2]{ati67}.

\section{Proof of the general case}

This corresponds to \cite[Proposition 2.7.8]{ati67}, \cite[Proposition 2.7.9]{ati67} and \cite[Proposition 2.7.12]{ati67}.

\appendix

\section{Conventions and preliminaries}

\subsection{Construction of $K(X)$}

The Grothendieck group is defined via universal property, but let us agree on a specific construction in order to have a precise description of the elements in the ring $K(X)$.
We follow both \cite{ati67} and \cite{hat03} and consider the construction $1$ described by Jin in the first talk of the seminar, which is the second construction discussed by Atiyah in \cite[p.~42]{ati67}.

\begin{lm}
  Let $M$ be a commutative monoid.
  Then Jin's construction $1$ agrees with Atiyah's second construction of $K(M)$.
  \begin{proof}
    In both cases $K(M)$ is the quotient of $M \times M$ by an equivalence relation, so it suffices to show that the equivalence relations agree.
    In Atiyah's construction we have
    \[ (x,y) \sim_{A} (x', y') : \Leftrightarrow \exists z , z' \in M,, (x + z, y + z) = (x' + z', y' + z'). \]
    In Jin's construction we have
    \[ (x,y) \sim_{J} (x', y') : \Leftrightarrow \exists z \in M,, x + y' + z = x' + y + z. \]
    If $(x, y) \sim_{A} (x', y')$, then we have $x + z = x' + z'$ and $y + z = y' + z'$ for some $z, z' \in M$. 
    Associativity and commutativity of $M$ imply that
    \[ x + y' + z + z' = x' + y' + z' + z' = x' + y + z + z', \]
    hence $(x, y) \sim_{J} (x', y')$.
    Conversely, if $(x, y) \sim_{J} (x', y')$, then we have $x + y' + z = x' + y + z$ for some $z \in M$.
    In particular we have
    \begin{align*}
      (x + (x + y' + z), y + (x + y' + z)) & = (x + (x' + y + z), y' + (x + y + z)) \\
      & = (x' + (x + y + z), y' + (x + y + z)),
    \end{align*}
    so $(x, y) \sim_{A} (x', y')$ as well.
  \end{proof}
\end{lm}

Given (an isomorphism class of) a vector bundle $E \in \operatorname{Vect}(X)$, we denote by $[E]$ its image in $K(X)$, that is, $[E] = [(E, 0)]$.
Since $-[E] = [(0,E)]$, we can write every element $[(E,F)] \in K(X)$ as $[E] - [F]$.
We can find some vector bundle $G$ such that $F \oplus G$ is trivial \cite[Corollary 1.4.14]{ati67}.
With the notation introduced earlier we can write $[F \oplus G] = [\underline{n}]$ for some $n \in \mathbb{N}$.
Then we would have
\[ [E] - [F] = [E] + [G] - ([F] + [G]) = [E \oplus G] - [\underline{n}], \]
showing that every element of $K(X)$ can be written as $[H] - [\underline{n}]$ for some vector bundle $H$ on $X$ and some natural number $n \in \mathbb{N}$ \cite[p.~44]{ati67}.

Suppose now that $E$ and $F$ are such that $[E] = [F]$, that is, $[(E,0)] = [(F,0)]$.
By definition of the equivalence relation that we are using, there exists some vector bundle $G$ such that $E \oplus G \cong F \oplus G$.
Applying \cite[Corollary 1.4.14]{ati67} again we deduce that $E \oplus \underline{n} \cong F \oplus \underline{n}$ for some $n \in \mathbb{N}$.
In this case we say that $E$ and $F$ are \textit{stably equivalent}.
This brings us to Hatcher's description of $K(X)$ \cite[p.~39]{hat03}, namely, as formal differences $E - E'$ in which we identify $E_{1} - E_{1}'$ with $E_{2} - E_{2}'$ if and only if $E_{1} \oplus E_{2}'$ and $E_{2} \oplus E_{1}'$ are stably equivalent, that is, if and only if $[E_{1} \oplus E_{2}'] = [E_{2} \oplus E_{1}']$.
Since $[E_{1} \oplus E_{2}'] = [E_{1}] + [E_{2}']$ and $[E_{2} \oplus E_{1}'] = [E_{2}] + [E_{1}']$, we do have $E_{1} - E_{1}' = E_{2} - E_{2}'$ in Hatcher's sense if and only if $[(E_{1},E_{1}')] = [(E_{2},E_{2}')]$ in Atiyah's sense.
We will try to follow Atiyah's notation most of the time.

\bibliographystyle{alpha}
\bibliography{main}
\vfill

\end{document}
