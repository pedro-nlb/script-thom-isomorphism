\documentclass[12pt,a4paper]{amsart}

\usepackage[T1]{fontenc}
\usepackage[utf8]{inputenc}
\usepackage[british]{babel}
\usepackage{mathtools}
\usepackage{amsthm}
\usepackage{amssymb}
\usepackage{mathrsfs}
\usepackage{enumitem}
\usepackage{tikz-cd}
\usetikzlibrary{decorations.markings}
\usepackage{float}
\usepackage{hyperref}
\urlstyle{same}
\usepackage[noabbrev]{cleveref}

\theoremstyle{plain}
\newtheorem{thm}{Theorem}
\newtheorem*{thm*}{Theorem}
\newtheorem{lm}[thm]{Lemma}
\newtheorem{prop}[thm]{Proposition}
\newtheorem{cor}[thm]{Corollary}
\theoremstyle{definition}
\newtheorem{defn}[thm]{Definition}
\newtheorem{exmp}[thm]{Example}
\newtheorem{xca}[thm]{Exercise}
\theoremstyle{remark}
\newtheorem{rem}[thm]{Remark}
\Crefname{thm}{Theorem}{Theorems}
\Crefname{lm}{Lemma}{Lemmas}
\Crefname{prop}{Proposition}{Propositions}
\Crefname{cor}{Corollary}{Corollaries}
\Crefname{defn}{Definition}{Definitions}
\Crefname{exmp}{Example}{Examples}
\Crefname{xca}{Exercise}{Exercises}
\Crefname{rem}{Remark}{Remarks}

\title[The Thom isomorphism]{The Thom isomorphism}
\author[Pedro N\'{u}\~{n}ez]{Pedro N\'{u}\~{n}ez}
\address{Pedro N\'{u}\~{n}ez \newline
\indent Albert-Ludwigs-Universit\"{a}t Freiburg, Mathematisches Institut \newline
\indent Ernst-Zermelo-Straße 1, 79104 Freiburg im Breisgau (Germany)}
\email{\normalfont\href{mailto:pedro.nunez@math.uni-freiburg.de}{pedro.nunez@math.uni-freiburg.de}}
\renewcommand*{\urladdrname}{\itshape Homepage}
\urladdr{\normalfont\href{https://home.mathematik.uni-freiburg.de/nunez/}{https://home.mathematik.uni-freiburg.de/nunez}}
\thanks{The author gratefully acknowledges support by the DFG-Graduiertenkolleg GK1821 ``Cohomological Methods in Geometry'' at the University of Freiburg.}
\date{\today}

\setcounter{tocdepth}{1}
\sloppy
\makeatletter
\hypersetup{
  pdfauthor={\authors},
  pdftitle={\@title},
  colorlinks,
  linkcolor=[rgb]{0.2,0.2,0.6},
  citecolor=[rgb]{0.2,0.6,0.2},
  urlcolor=[rgb]{0.6,0.2,0.2}}
\makeatother
\makeatletter
\tikzcdset{
  open/.code={\tikzcdset{hook, circled};},
  closed/.code={\tikzcdset{hook, slashed};},
  circled/.code={\tikzcdset{markwith={\draw (0,0) circle (.375ex);}};},
  slashed/.code={\tikzcdset{markwith={\draw[-] (-.4ex,-.4ex) -- (.4ex,.4ex);}};},
  markwith/.code={
    \pgfutil@ifundefined{tikz@library@decorations.markings@loaded}%
    {\pgfutil@packageerror{tikz-cd}{You need to say %
    \string\usetikzlibrary{decorations.markings} to use arrow with markings}{}}{}%
    \pgfkeysalso{/tikz/postaction={/tikz/decorate,
    /tikz/decoration={
      markings,
      mark = at position 0.5 with
      {#1}}}}},
}
\makeatother

\begin{document}

\maketitle

\begin{abstract}
  Script for a talk of the Wednesday Seminar of the GK1821 at Freiburg during the Summer Semester 2021.
  The main reference is \cite[\S 2]{ati67}.
\end{abstract}

\tableofcontents

\begin{center}
  \textcolor{gray}{---parts in gray will be omitted during the talk---}
\end{center}

\section{Recollection from previous talks}

We start off with some recollections from \cite[\S 2]{ati67}.
We refer to \cite{ati67} and the appendix for previous conventions and other preliminaries.

\subsection{The functor $K(-)$ \cite[p.~44]{ati67}}

Let $\mathbf{CHaus}$ be the category of compact Hausdorff topological spaces and let $\mathbf{Ring}$ be the category of commutative unital rings.
We have defined a functor

\[ K \colon \mathbf{CHaus} \to \mathbf{Ring} \]

which can be explicitly described as follows.
If $E \in \operatorname{Vect}(X)$ is a vector bundle on a compact Hausdorff space $X$, then we denote by $[E]$ its stable equivalence class, i.e.,

\[ [E] := \{ F \in \operatorname{Vect}(X) \mid \exists n \in \mathbb{N} \text{ s.t.~} E \oplus n \cong F \oplus n \}, \]

where we denote also by $n$ the trivial vector bundle of rank $n$ on $X$.
Then the underlying set of the ring $K(X)$ consists of formal differences $[E] - [F]$ for vector bundles $E, F \in \operatorname{Vect}(X)$.
The sum is given by
\[ ([E] - [F]) + ([E'] - [F']) = [E \oplus E'] - [F \oplus F'], \]
and the product is given by
\[ ([E] - [F]) ([E'] - [F']) = [(E \otimes E') \oplus (F \otimes F')] - [(E \otimes F') \oplus (E' \otimes F)]. \]
The element zero can be represented by $[0] - [0]$ and the element one by $[1] - [0]$, where again by $0$ we mean $X \times \{0\}$ and by $1$ we mean $X \times \mathbb{C}$.
Moreover, since $X$ is compact and Hausdorff, we can represent every element of $K(X)$ as $[E] - [n]$ for some $E \in \operatorname{Vect}(X)$ and some $n \in \mathbb{N}$.

If $f \colon X \to Y$ is a continuous map between compact Hausdorff topological spaces, then $K(f) =: f^{*}$ is given by

\begin{align*}
  f^{*} \colon K(Y) & \to K(X) \\
  [E] - [F] & \mapsto [f^{*}(E)] - [f^{*}(F)],
\end{align*}

and this ring homomorphism only depends on the homotopy class of $f$.

\subsection{The functor $\tilde{K}(-)$ \cite[p.~66]{ati67}}

Let $\mathbf{CHaus}^{+}$ be the category of pointed compact Hausdorff topological spaces and let $\mathbf{Rng}$ be the category of commutative non-unital rings.
We have defined a functor

\[ \tilde{K} \colon \mathbf{CHaus}^{+} \to \mathbf{Rng} \]

as follows.
Let $(X,x_{0})$ be pointed compact Hausdorff topological space and let $i \colon \{ x_{0} \} \to X$ denote the inclusion of the base point.

\appendix

\section{Conventions and preliminaries}

\subsection{Construction of $K(X)$}

The Grothendieck group is defined via universal property, but let us agree on a specific construction in order to have a precise description of the elements in the ring $K(X)$.
We follow both \cite{ati67} and \cite{hat03} and consider the construction $1$ described by Jin in the first talk of the seminar, which is the second construction discussed by Atiyah in \cite[p.~42]{ati67}.

\begin{lm}
  Let $M$ be a commutative monoid.
  Then Jin's construction $1$ agrees with Atiyah's second construction of $K(M)$.
  \begin{proof}
    In both cases $K(M)$ is the quotient of $M \times M$ by an equivalence relation, so it suffices to show that the equivalence relations agree.
    In Atiyah's construction we have
    \[ (x,y) \sim_{A} (x', y') : \Leftrightarrow \exists z , z' \in M,, (x + z, y + z) = (x' + z', y' + z'). \]
    In Jin's construction we have
    \[ (x,y) \sim_{J} (x', y') : \Leftrightarrow \exists z \in M,, x + y' + z = x' + y + z. \]
    If $(x, y) \sim_{A} (x', y')$, then we have $x + z = x' + z'$ and $y + z = y' + z'$ for some $z, z' \in M$. 
    Associativity and commutativity of $M$ imply that
    \[ x + y' + z + z' = x' + y' + z' + z' = x' + y + z + z', \]
    hence $(x, y) \sim_{J} (x', y')$.
    Conversely, if $(x, y) \sim_{J} (x', y')$, then we have $x + y' + z = x' + y + z$ for some $z \in M$.
    In particular we have
    \begin{align*}
      (x + (x + y' + z), y + (x + y' + z)) & = (x + (x' + y + z), y' + (x + y + z)) \\
      & = (x' + (x + y + z), y' + (x + y + z)),
    \end{align*}
    so $(x, y) \sim_{A} (x', y')$ as well.
  \end{proof}
\end{lm}

Given (an isomorphism class of) a vector bundle $E \in \operatorname{Vect}(X)$, we denote by $[E]$ its image in $K(X)$, that is, $[E] = [(E, 0)]$.
Since $-[E] = [(0,E)]$, we can write every element $[(E,F)] \in K(X)$ as $[E] - [F]$.
We can find some vector bundle $G$ such that $F \oplus G$ is trivial \cite[Corollary 1.4.14]{ati67}.
With the notation introduced earlier we can write $[F \oplus G] = [\underline{n}]$ for some $n \in \mathbb{N}$.
Then we would have
\[ [E] - [F] = [E] + [G] - ([F] + [G]) = [E \oplus G] - [\underline{n}], \]
showing that every element of $K(X)$ can be written as $[H] - [\underline{n}]$ for some vector bundle $H$ on $X$ and some natural number $n \in \mathbb{N}$ \cite[p.~44]{ati67}.

Suppose now that $E$ and $F$ are such that $[E] = [F]$, that is, $[(E,0)] = [(F,0)]$.
By definition of the equivalence relation that we are using, there exists some vector bundle $G$ such that $E \oplus G \cong F \oplus G$.
Applying \cite[Corollary 1.4.14]{ati67} again we deduce that $E \oplus \underline{n} \cong F \oplus \underline{n}$ for some $n \in \mathbb{N}$.
In this case we say that $E$ and $F$ are \textit{stably equivalent}.
This brings us to Hatcher's description of $K(X)$ \cite[p.~39]{hat03}, namely, as formal differences $E - E'$ in which we identify $E_{1} - E_{1}'$ with $E_{2} - E_{2}'$ if and only if $E_{1} \oplus E_{2}'$ and $E_{2} \oplus E_{1}'$ are stably equivalent, that is, if and only if $[E_{1} \oplus E_{2}'] = [E_{2} \oplus E_{1}']$.
Since $[E_{1} \oplus E_{2}'] = [E_{1}] + [E_{2}']$ and $[E_{2} \oplus E_{1}'] = [E_{2}] + [E_{1}']$, we do have $E_{1} - E_{1}' = E_{2} - E_{2}'$ in Hatcher's sense if and only if $[(E_{1},E_{1}')] = [(E_{2},E_{2}')]$ in Atiyah's sense.
We will try to follow Atiyah's notation most of the time.

\bibliographystyle{alpha}
\bibliography{main}
\vfill

\end{document}
