\documentclass[12pt,a4paper]{amsart}

\usepackage[T1]{fontenc}
\usepackage[utf8]{inputenc}
\usepackage[british]{babel}
\usepackage{mathtools}
\usepackage{amsthm}
\usepackage{amssymb}
\usepackage{mathrsfs}
\usepackage{enumitem}
\usepackage{tikz-cd}
\usetikzlibrary{decorations.markings}
\usepackage{float}
\usepackage{hyperref}
\urlstyle{same}
\usepackage[noabbrev]{cleveref}

\theoremstyle{plain}
\newtheorem{thm}{Theorem}
\newtheorem*{thm*}{Theorem}
\newtheorem{lm}[thm]{Lemma}
\newtheorem{prop}[thm]{Proposition}
\newtheorem{cor}[thm]{Corollary}
\theoremstyle{definition}
\newtheorem{defn}[thm]{Definition}
\newtheorem{exmp}[thm]{Example}
\newtheorem{xca}[thm]{Exercise}
\theoremstyle{remark}
\newtheorem{rem}[thm]{Remark}
\Crefname{thm}{Theorem}{Theorems}
\Crefname{lm}{Lemma}{Lemmas}
\Crefname{prop}{Proposition}{Propositions}
\Crefname{cor}{Corollary}{Corollaries}
\Crefname{defn}{Definition}{Definitions}
\Crefname{exmp}{Example}{Examples}
\Crefname{xca}{Exercise}{Exercises}
\Crefname{rem}{Remark}{Remarks}

\title[The Thom isomorphism]{The Thom isomorphism}
\author[Pedro N\'{u}\~{n}ez]{Pedro N\'{u}\~{n}ez}
\address{Pedro N\'{u}\~{n}ez \newline
\indent Albert-Ludwigs-Universit\"{a}t Freiburg, Mathematisches Institut \newline
\indent Ernst-Zermelo-Straße 1, 79104 Freiburg im Breisgau (Germany)}
\email{\normalfont\href{mailto:pedro.nunez@math.uni-freiburg.de}{pedro.nunez@math.uni-freiburg.de}}
\renewcommand*{\urladdrname}{\itshape Homepage}
\urladdr{\normalfont\href{https://home.mathematik.uni-freiburg.de/nunez/}{https://home.mathematik.uni-freiburg.de/nunez}}
\thanks{The author gratefully acknowledges support by the DFG-Graduiertenkolleg GK1821 ``Cohomological Methods in Geometry'' at the University of Freiburg.}
\date{30th June 2021}

\setcounter{tocdepth}{1}
\sloppy
\makeatletter
\hypersetup{
  pdfauthor={\authors},
  pdftitle={\@title},
  colorlinks,
  linkcolor=[rgb]{0.2,0.2,0.6},
  citecolor=[rgb]{0.2,0.6,0.2},
  urlcolor=[rgb]{0.6,0.2,0.2}}
\makeatother

\begin{document}

\maketitle

\begin{abstract}
  Script for a talk of the Wednesday Seminar of the GK1821 at Freiburg during the Summer Semester 2021.
  The main reference is \cite[\S 2]{ati67}.
\end{abstract}

\tableofcontents

\section{Recollection from previous talks}

We start off with some recollections from \cite[\S 2]{ati67}.

\subsection{Notation and topological preliminaries}

For $n \in \mathbb{N}$ we denote by $S^{n}$ the $n$-dimensional sphere.
We think of it as the pointed compact Hausdorff space $I^{n}/\partial I^{n}$, where $I^{n} \subseteq \mathbb{R}^{n}$ is the unit cube and the basepoint is given by the equivalence class of any point in the boundary $\partial I^{n}$.
The unit interval $I^{1} = [0,1]$ is denoted simply by $I$.

Given pointed compact Hausdorff spaces $(X,x_{0})$ and $(Y,y_{0})$, we define their \textit{smash product} as
\[ X \wedge Y := X \times Y / X \vee Y, \]
where $X \vee Y := X \times \{y_{0}\} \cup \{ x_{0}\} \times Y$ is their \textit{wedge sum}.
The \textit{reduced suspension} of $(X,x_{0})$ is obtained from the usual suspension $(I\times X)/(\{0\}\times X \cup \{1\} \times X)$ by further collapsing the line $I \times \{ x_{0} \}$ joining the two vertices of the suspension along the basepoint.
Using our explicit description of $S^{1}$, we may rewrite the reduced suspension of $X$ as $S^{1} \wedge X$, and we denote it by $SX$.
The $n$-th iterated suspension\footnote{Smash product is associative on compact spaces, but not in general \color{red} elaborate! \color{black}.} is naturally homeomorphic to $S^{n} \wedge X$, and we denote it by $S^{n}X$. % TODO
Finally, the \textit{cone} of the pointed compact Hausdorff space $(X,x_{0})$ is defined as $CX := (I \times X )/ (\{0 \} \times X)$, and we take as basepoint the equivalence class of any point in the subspace that we are collapsing \cite[p.~68]{ati67}.

Following \cite[p.~66]{ati67}, we denote by $\mathcal{C}$ the category of compact Hausdorff spaces, by $\mathcal{C}^{+}$ the category of pointed compact Hausdorff spaces and by $\mathcal{C}^{2}$ the category consisting of pairs $(X,Y)$ where $X$ and $Y$ are compact Hausdorff spaces and $Y \subseteq X$ is a subspace.
Equivalently, $X$ is a compact Hausdorff space and $Y \subseteq X$ is a closed subset.

We will consider complex vector bundles over compact Hausdorff spaces.
If $E$ is a vector bundle over $X$, then we will often abuse notation and denote again by $E$ its isomorphism class.
We will denote by $p \colon P(E) \to X$ the \textit{projective bundle associated to $E$}, by $H^{*} \subseteq p^{*}E$ the \textit{tautological line bundle} and by $H := (H^{*})^{*}$ its dual, which we will call the \textit{hyperplane bundle} \cite[p.~45]{ati67}.

\subsection{The functor $\operatorname{Vect}(-)$}

For a compact Hausdorff space $X$, we denote by $\operatorname{Vect}(X)$ the set of isomorphism classes of vector bundles on $X$.
This set becomes a commutative monoid with respect to direct sum of vector bundles; the zero vector bundle is the neutral element of this monoid.
Moreover, tensor product of vector bundles induces a multiplication which turns this commutative monoid into a commutative semiring; the trivial line bundle is the unit in this semiring.

\begin{rem}
  We need to talk about isomorphism classes of vector bundles---and not just about vector bundles---in order to ensure that the axioms of a commutative semiring are satisfied, e.g.~the addition would not be commutative otherwise.
\end{rem}

If $f \colon X \to Y$ is a continuous function between compact Hausdorff spaces, then the pullback along $f$ defines a morphism
\[ f^{*} \colon \operatorname{Vect}(Y) \to \operatorname{Vect}(X) \]
of semirings with unit.
We have $\operatorname{id}_{X}^{*} = \operatorname{id}_{\operatorname{Vect}(X)}$ and $(g \circ f)^{*} = f^{*} \circ g^{*}$, so we obtain a functor
\[ \operatorname{Vect} \colon \mathcal{C}^{opp} \to \mathbf{SemiRing}, \]
where $\mathbf{SemiRing}$ denotes the category of commutative semirings with unit and $(-)^{opp}$ denotes the opposite category.

\begin{rem}
  Again, we need to talk about isomorphism classes of vector bundles in order to ensure that the axioms of a functor are satisfied, e.g.~the composition would not be sent to the composition otherwise.
\end{rem} 

\subsection{The functor $K(-)$}
We follow \cite[p.~44]{ati67} in this subsection.
The Grothendieck group construction yields a functor $G \colon \mathbf{SemiRing} \to \mathbf{Ring}$, where $\mathbf{Ring}$ denotes the category of commutative rings with unit.
The ring $G(A)$ corresponding to a semiring $A$ can be described in various ways; let us fix one description for concreteness.
As an abelian group, $G(A)$ is given by $A \times A/\Delta(A)$, where $\Delta \colon A \to A \times A$ is the diagonal morphism.
Thus, the set $G(A)$ consists of equivalence classes of pairs $(a,b) \in A \times A$, where $(a,b) \sim (a',b')$ if and only if there exists $z, z' \in A$ such that
\[ (a + z, b + z) = (a' + z', b' + z'). \]
The addition in $G(A)$ is given by $[(a,b)] + [(a',b')] = [(a + a', b + b')]$.
The neutral element of the group $G(A)$ is then the equivalence class of $(a,a)$ for any $a \in A$, and the inverse of $[(a,b)]$ is $[(b,a)]$.
We can think of $[(a,b)]$ as $a - b$.
More precisely, if for each $a \in A$ we denote by $[a]$ its image in $G(A)$, i.e.~$[a] := [(a,0)]$, then $[(a,b)] = [a] - [b]$.
With this in mind we see that the induced multiplication in $G(A)$ is given by
\[ [(a,b)][(a',b')] = [(aa'+bb', ab' + a'b)], \]
and the unit of the ring $G(A)$ is $[(1_{A},0)]$.
The natural monoid morphism $A \to G(A)$ given by $a \mapsto [a]$ is then also a morphism of semirings with unit.
And if $\varphi \colon A \to B$ is a morphism of semirrings with unit, then the induced $G(\varphi) \colon G(A) \to G(B)$ given by $[(a,b)] \mapsto [(\varphi(a), \varphi(b))]$ is a morphism of commutative rings with unit.
Thus we have a functor $G \colon \mathbf{SemiRing} \to \mathbf{Ring}$ as claimed.
The functor
\[ K \colon \mathcal{C}^{opp} \to \mathbf{Ring} \]
is then defined as the composition $G \circ \operatorname{Vect}$.

To make things more explicit, let $X$ be a compact Hausdorff space.
Then an element of the ring $K(X)$ is an equivalence class $[(E,F)]$ where $E$ and $F$ are (isomorphism classes of) vector bundles on $X$.
As explained above, we may rewrite this element as $[E] - [F]$.
The addition in $K(X)$ is then given by
\[ ([E] - [F]) + ([E'] - [F']) = [E \oplus E'] - [F \oplus F'], \]
and the product is given by
\[ ([E] - [F]) ([E'] - [F']) = [(E \otimes E') \oplus (F \otimes F')] - [(E \otimes F') \oplus (E' \otimes F)]. \]
For $n \in \mathbb{N}$, we denote by $[n]$ the image of $X \times \mathbb{C}^{n}$ in $K(X)$.
The element zero can be represented by $[0] - [0]$ and the element one by $[1] - [0]$, where again by $0$ we mean $X \times \{0\}$ and by $1$ we mean $X \times \mathbb{C}$.
Moreover, since $X$ is compact and Hausdorff, we can represent every element of $K(X)$ as $[E] - [n]$ for some $E \in \operatorname{Vect}(X)$ and some $n \in \mathbb{N}$ \cite[p.~44]{ati67}.

If $f \colon X \to Y$ is a continuous map between compact Hausdorff topological spaces, then $K(f) =: f^{*}$ is given by

\begin{align*}
  f^{*} \colon K(Y) & \to K(X) \\
  [E] - [F] & \mapsto [f^{*}(E)] - [f^{*}(F)],
\end{align*}

and this ring morphism only depends on the homotopy class of $f$.

\subsection{The functor $\tilde{K}(-)$}
We follow \cite[p.~66]{ati67} in this subsection.
Let $\mathbf{Rng}$ be the category of commutative non-unital rings.
We define a functor $\tilde{K} \colon \mathcal{C}^{+} \to \mathbf{Rng}$ as follows.
Let $(X,x_{0})$ be a pointed compact Hausdorff topological space and let $i \colon \{ x_{0} \} \to X$ denote the inclusion of the basepoint.
Then
\[ \tilde{K}(X) := \ker(i^{*}) \subseteq K(X). \]

\begin{rem}
  Let $\xi = [E] - [F] \in K(X)$.
  Then $\xi \in \tilde{K}(X)$ if and only if $\dim(E_{x_{0}}) = \dim(F_{x_{0}})$.
  Indeed, it suffices to show that the difference $\dim(E_{x_{0}}) - \dim(F_{x_{0}})$ does not depend on the representative $[E] - [F]$ of the equivalence class $\xi$.
  But if $[(E,F)] = [(E',F')]$, then there exist $G,G' \in \operatorname{Vect}(X)$ such that $E \oplus G = E' \oplus G'$ and $F \oplus G = F' \oplus G'$.
  Hence $\dim(E_{x_{0}}) - \dim(F_{x_{0}}) = \dim(E'_{x_{0}}) - \dim(F'_{x_{0}})$.
\end{rem}

If $c \colon X \to \{ x_{0} \}$ is the constant morphism to the basepoint, then $c^{*}$ induces a splitting $K(X) \cong \tilde{K}(X) \oplus K(x_{0})$ which is natural with respect to morphisms in $\mathcal{C}^{+}$.
Hence $\tilde{K}(-)$ is a functor as claimed above.
If $f \colon (X,x_{0}) \to (Y,y_{0})$ is a morphism in $\mathcal{C}^{+}$ given by a continuous function $f \colon X \to Y$ such that $f(x_{0}) = y_{0}$, then $\tilde{K}(f) =: f^{*}$ is induced by the restriction of $f^{*} \colon K(Y) \to K(X)$.
One can also check directly that $f^{*} \colon K(Y) \to K(X)$ induces $f^{*} \colon \tilde{K}(Y) \to \tilde{K}(X)$ using the previous remark and the definition of pullbacks of vector bundles.

\begin{rem}
  We can recover the functor $K(-)$ from the functor $\tilde{K}(-)$ by adding disjoint basepoints.
  More precisely, for any compact Hausdorff space $X$, there is a ring isomorphism
  \begin{align*} 
    K(X) & \to \tilde{K}(X^{+}) \\
    [E] - [F] & \mapsto [E \sqcup 0] - [F \sqcup 0]
  \end{align*}
  where by $0$ we mean $\{ x_{0} \} \times \mathbb{C}^{0}$, so that $E \sqcup 0$ is a vector bundle over $X^{+} := X \sqcup \{ x_{0} \}$.
\end{rem}

\begin{rem}
  As the previous remark already suggests, $\tilde{K}(X^{+})$ is a unital ring for all non-empty $X$.
  Its unit is given by the image of the unit in $K(X)$ under the previous ring isomorphism, i.e.~it is given by $[1 \sqcup 0] - [0 \sqcup 0]$.
  But if $(X, x_{0})$ is a connected pointed compact Hausdorff space, then $\tilde{K}(X)$ cannot have a unit $1 \neq 0$, because every element in $\tilde{K}(X)$ is nilpotent \cite[Theorem II.5.9]{kar78}.
\end{rem}

\subsection{The functor $K(-,-)$}
We follow \cite[p.~66]{ati67} in this subsection.
We define a functor $K \colon \mathcal{C}^{2} \to \mathbf{Rng}$ as follows.
For an object $(X,Y)$ in $\mathcal{C}^{2}$, we set
\[ K(X,Y) := \tilde{K}(X/Y), \]

where the basepoint of $X/Y$ is the equivalence class of any point $y \in Y$, which we denote by $Y/Y$.
If $f \colon (X,Y) \to (Z,W)$ is a morphism in $\mathcal{C}^{2}$ given by a continuous function $f \colon X \to Z$ such that $f(Y) \subseteq W$, then it induces a morphism $\bar{f} \colon (X/Y,Y/Y) \to (Z/W, W/W)$ in $\mathcal{C}^{+}$.
Then $K(f) =: f^{*}$ is given by $\tilde{K}(\bar{f})$.

\begin{rem}
  We can recover the functor $K(-)$ from the functor $K(-,-)$ by considering the empty subspace.
  More precisely, for any compact Hausdorff space $X$, using our convention\footnote{For $Y \subseteq X$ we may define $X/Y$ as the pushout $X \sqcup_{Y} \{ * \}$.
  Then we really have $X/\varnothing = X \sqcup \{ * \}$.} that $X/\varnothing = X^{+}$, we have again the ring isomorphism discussed above
  \[ K(X) \to \tilde{K}(X^{+}) =: K(X, \varnothing). \]
\end{rem}

\subsection{The six-term exact sequence}

\begin{defn}[{\cite[Definition 2.4.1]{ati67}}]
  For $n \in \mathbb{N}_{>0}$ we define a functor $\tilde{K}^{-n} \colon \mathcal{C}^{+} \to \mathbf{Rng}$ by setting
  \[ \tilde{K}^{-n}(X) := \tilde{K}(S^{n}X). \]
  Then we define a functor $K^{-n} \colon \mathcal{C}^{2} \to \mathbf{Rng}$ via
  \[ K^{-n}(X,Y) := \tilde{K}^{-n}(X/Y), \]
  and finally a functor $K^{-n} \colon \mathcal{C} \to \mathbf{Rng}$ with the formula
  \[ K^{-n}(X) := K^{-n}(X, \varnothing). \]
\end{defn}

\begin{rem}
  For $n = 0$ we have seen the relations
  \[ K^{0}(-) = \tilde{K}^{0}((-)^{+}) = K^{0}(-, \varnothing). \]
  Let now $n \in \mathbb{N}$ be arbitrary.
  Then we still have $K^{-n}(-) = K^{-n}(-, \varnothing)$ by definition.
  We also have $K^{-n}(-, \varnothing) = \tilde{K}^{-n}((-)^{+})$ by definition, so we still have the same relations
  \[ K^{-n}(-) = \tilde{K}^{-n}((-)^{+}) = K^{-n}(-, \varnothing). \]
\end{rem}

\begin{rem}
  Let $(X,x_{0})$ be a pointed compact Hausdorff space.
  Then we have a canonical group isomorphism $K^{-1}(X) \cong \tilde{K}^{-1}(X)$.
  Unlike in the case of singular homology and its reduced counterpart, the existence of this isomorphism is not immediate.
  For example, this is already non-trivial in the case of a point.
  By definition we have $K^{-1}(\{ x \}) = \tilde{K}(S(\{ x \} \sqcup \{ x_{0} \})) = \tilde{K}(S^{1})$.
  The isomorphism is then true despite the spaces being so different, because $\tilde{K}(\{ x \}) = \tilde{K}(S^{1}) = 0$.
  The general case follows from the isomorphisms
  \[ \tilde{K}(S(X^{+})) = \tilde{K}(S(X) \vee S^{1}) \cong \tilde{K}(SX) \oplus \tilde{K}(S^{1}) \cong \tilde{K}(SX), \]
  see \cite[p.~57]{hat03}.
\end{rem}

Recall from Vera's talk:

\begin{prop}[{\cite[Proposition 2.4.4]{ati67}}]
  For each pair $(X,Y)$ in $\mathcal{C}^{2}$ there is a natural exact sequence
  \[ \cdots \to K^{-1}(Y) \xrightarrow{\delta} K^{0}(X,Y) \xrightarrow{j^{*}} K^{0}(X) \xrightarrow{i^{*}} K^{0}(Y), \]
  where $i \colon Y \to X$ and $j \colon (X,\varnothing) \to (X,Y)$ are the inclusions.
\end{prop}

\begin{rem}[{\cite[p.~87]{ati67}}]
  The morphism $\delta \colon K(S(Y),\varnothing) \to K(X,Y)$ is also induced by a morphism in $\mathcal{C}^{2}$, namely, $\delta = \iota^{*}$ for the inclusion
  \[ \iota \colon (\{1\} \times X \cup I \times Y, \{0\} \times Y ) \hookrightarrow (\{1 \} \times X \cup I \times Y, \{ 0\} \times Y \cup \{1 \} \times X), \]
  modulo some topology allowing us to identify the reduced $K$-groups of the corresponding quotients with the reduced $K$-groups that we want.
\end{rem}

Using the periodicity ringisomorphisms $\beta \colon K^{-n}(X) \to K^{-n-2}(X)$ from \cite[Theorem 2.4.9]{ati67}, or rather their reduced version $\beta \colon \tilde{K}^{-n}(X) \to \tilde{K}^{-n-2}(X)$, we extend the collection of functors $\{ K^{-n}(-,-) \}_{n \in \mathbb{N}}$ to a collection of funcotrs $\{ K^{n}(-,-) \}_{n \in \mathbb{Z}}$ inductively, i.e.~$K^{1} = K^{-1}$, $K^{2} = K^{0}$, $K^{3} =K^{1} = K^{-1}$, and so on.
Then we use the previously seen relations among the different $K$ functors to define collections of functors $\{ K^{n}(-) \}_{n \in \mathbb{Z}}$ and $\{ \tilde{K}^{n}(-) \}_{n \in \mathbb{Z}}$ as well.
This allows us to extend the previous exact sequence to the right, and identifying $K^{n}$ with $K^{n+2}$ for all $n \in \mathbb{Z}$ we may rearrange this data into the following \textit{six-term exact sequence}

\begin{center}
  \begin{tikzcd}
    K^{0}(X,Y) \arrow{r} & K^{0}(X) \arrow{r} & K^{0}(Y) \arrow{d} \\
    K^{1}(Y) \arrow{u} & K^{1}(X) \arrow{l} & K^{1}(X,Y) \arrow{l}
  \end{tikzcd}
\end{center}

\section{The functor $K^{*}(-,-)$}

\begin{defn}
  Let $X$ be a compact Hausdorff space.
  Then we define
  \[ K^{*}(X) := K^{0}(X) \oplus K^{1}(X). \]
  Similarly, for a pair $(X,Y)$ in $\mathcal{C}^{2}$ we define
  \[ K^{*}(X,Y) := K^{0}(X,Y) \oplus K^{1}(X,Y) \]
  and finally
  \[ \tilde{K}^{*}(X) := K^{*}(X,\varnothing). \]
\end{defn}

\begin{rem}
  % TODO finish
  Ring structure on $K^{*}$ \color{red} finish! \color{black}
  In fact, we get the triangle of \cite[p.~87]{ati67} in which all morphisms are $K^{*}(X)$-module morphisms.
\end{rem}

\section{Thom spaces and statement of the theorem}

Introduction and goal of the section: introduce Thom spaces and state the Thom isomorphism theorem.

Brief spoiler defining Thom spaces here already.

\subsection{Even dimensional spheres}

Recall the structure of the group $\tilde{K}^{0}(\mathbb{S}^{2n})$ determined in Vera's talk \cite[Corollary 2.12]{hat03}.
We already have a generator from this result in Hatcher's book.
We can describe this canonical generator---up to a sign---in terms of the exterior algebra on an $n$-dimensional $\mathbb{C}$-vector space $V$ \cite[p.~99]{ati67}.

\subsection{Thom spaces}

Generalize discussion on even dimensional spheres to Thom spaces as in \cite[p.~100]{ati67}.
Define the canonical $\lambda_{E} \in \tilde{K}(X^{E})$ and explain its properties.
Explain this also in terms of projectivizations of vector bundles plus a trivial line bundle, and recall Prof.~Huber's remark that this projectivization is a way to compactify the vector bundle.

\subsection{Statement of the theorem}

State the Thom isomorphism theorem \cite[Corollary 2.7.12]{ati67}.

\section{Proof for sums of line bundles}

In this section we work with a direct sum of line bundles

\[ E = L_{1} \oplus \ldots \oplus L_{m}. \]

\subsection{Recall $K^{0}(\mathbb{P}(E))$}

This is treated in \cite[Proposition 2.5.3]{ati67}; leave proof in gray, omitted during the talk.

\subsection{Computation of $K^{*}(\mathbb{P}(E))$}

This is \cite[Proposition 2.7.1]{ati67}.

\subsection{Thom isomorphism theorem in this case}

This is \cite[Proposition 2.7.2]{ati67}.

\section{Proof of the general case}

This corresponds to \cite[Proposition 2.7.8]{ati67}, \cite[Proposition 2.7.9]{ati67} and \cite[Proposition 2.7.12]{ati67}.

\appendix

\section{Conventions and preliminaries}

\subsection{Construction of $K(X)$}

The Grothendieck group is defined via universal property, but let us agree on a specific construction in order to have a precise description of the elements in the ring $K(X)$.
We follow both \cite{ati67} and \cite{hat03} and consider the construction $1$ described by Jin in the first talk of the seminar, which is the second construction discussed by Atiyah in \cite[p.~42]{ati67}.

\begin{lm}
  Let $M$ be a commutative monoid.
  Then Jin's construction $1$ agrees with Atiyah's second construction of $K(M)$.
  \begin{proof}
    In both cases $K(M)$ is the quotient of $M \times M$ by an equivalence relation, so it suffices to show that the equivalence relations agree.
    In Atiyah's construction we have
    \[ (x,y) \sim_{A} (x', y') : \Leftrightarrow \exists z , z' \in M,, (x + z, y + z) = (x' + z', y' + z'). \]
    In Jin's construction we have
    \[ (x,y) \sim_{J} (x', y') : \Leftrightarrow \exists z \in M,, x + y' + z = x' + y + z. \]
    If $(x, y) \sim_{A} (x', y')$, then we have $x + z = x' + z'$ and $y + z = y' + z'$ for some $z, z' \in M$. 
    Associativity and commutativity of $M$ imply that
    \[ x + y' + z + z' = x' + y' + z' + z' = x' + y + z + z', \]
    hence $(x, y) \sim_{J} (x', y')$.
    Conversely, if $(x, y) \sim_{J} (x', y')$, then we have $x + y' + z = x' + y + z$ for some $z \in M$.
    In particular we have
    \begin{align*}
      (x + (x + y' + z), y + (x + y' + z)) & = (x + (x' + y + z), y' + (x + y + z)) \\
      & = (x' + (x + y + z), y' + (x + y + z)),
    \end{align*}
    so $(x, y) \sim_{A} (x', y')$ as well.
  \end{proof}
\end{lm}

Given (an isomorphism class of) a vector bundle $E \in \operatorname{Vect}(X)$, we denote by $[E]$ its image in $K(X)$, that is, $[E] = [(E, 0)]$.
Since $-[E] = [(0,E)]$, we can write every element $[(E,F)] \in K(X)$ as $[E] - [F]$.
We can find some vector bundle $G$ such that $F \oplus G$ is trivial \cite[Corollary 1.4.14]{ati67}.
With the notation introduced earlier we can write $[F \oplus G] = [\underline{n}]$ for some $n \in \mathbb{N}$.
Then we would have
\[ [E] - [F] = [E] + [G] - ([F] + [G]) = [E \oplus G] - [\underline{n}], \]
showing that every element of $K(X)$ can be written as $[H] - [\underline{n}]$ for some vector bundle $H$ on $X$ and some natural number $n \in \mathbb{N}$ \cite[p.~44]{ati67}.

Suppose now that $E$ and $F$ are such that $[E] = [F]$, that is, $[(E,0)] = [(F,0)]$.
By definition of the equivalence relation that we are using, there exists some vector bundle $G$ such that $E \oplus G \cong F \oplus G$.
Applying \cite[Corollary 1.4.14]{ati67} again we deduce that $E \oplus \underline{n} \cong F \oplus \underline{n}$ for some $n \in \mathbb{N}$.
In this case we say that $E$ and $F$ are \textit{stably equivalent}.
This brings us to Hatcher's description of $K(X)$ \cite[p.~39]{hat03}, namely, as formal differences $E - E'$ in which we identify $E_{1} - E_{1}'$ with $E_{2} - E_{2}'$ if and only if $E_{1} \oplus E_{2}'$ and $E_{2} \oplus E_{1}'$ are stably equivalent, that is, if and only if $[E_{1} \oplus E_{2}'] = [E_{2} \oplus E_{1}']$.
Since $[E_{1} \oplus E_{2}'] = [E_{1}] + [E_{2}']$ and $[E_{2} \oplus E_{1}'] = [E_{2}] + [E_{1}']$, we do have $E_{1} - E_{1}' = E_{2} - E_{2}'$ in Hatcher's sense if and only if $[(E_{1},E_{1}')] = [(E_{2},E_{2}')]$ in Atiyah's sense.
We will try to follow Atiyah's notation most of the time.

\bibliographystyle{alpha}
\bibliography{main}
\vfill

\end{document}
